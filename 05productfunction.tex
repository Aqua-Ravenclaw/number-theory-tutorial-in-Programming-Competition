\chapter{积性函数}

\begin{introduction}
	\item xxx
\end{introduction}


\section{因子次幂和函数}


\section{欧拉函数}


\section{莫比乌斯函数}


\section{莫比乌斯反演}


\section{积性函数与狄利克雷卷积}


\section{积性函数前缀和}


\section{杜教筛}


\section{拓展埃氏筛}
拓展埃氏筛(Extended Eratosthenes Sieve)似乎最早由min25引入竞赛圈(因此也常被称为min25筛),可以在{\heiti 低于线性的时间}内求得积性函数(一些非积性函数也可以)的前缀和。
由于其方法简单且灵活性高,近年来经常在算法竞赛中出现,下面就来介绍一下EES。

首先要介绍的是在EES中要使用到的一个前置算法,叫做The Meissel, Lehmer, Lagarias, Miller, Odlyzko Method{\heiti (MLLMO Method)},论文\cite{Deleglise1996Computing}中
对其进行了系统介绍,有兴趣的可以继续阅读。

\subsection{MLLMO Method}
MLLMO Method要解决的是这样一个问题:
求解
$$
\sum_{i=2}^n[i\in Prime]F(i)
$$
其中$F(x) = x^k$,即求解
$$
\sum_{i=2}^n[i\in Prime]\ i^k
$$
设$P_j$表示第$j$个质数,$P_1=2,\ P_2=3\ ...$,特殊地$P_0$可以认为是$0$。

MLLMO Method用动态规划的思想解决这个问题:

设
$$
dp(n,j)=\sum_{i=2}^ni^k\ [\ i\in Prime\ or \ min(p)>P_j,\ where\ p|i\ and\ p\in Prime\ ]
$$
也就是说$i$是质数,或者合数$i$的最小质因子大于$P_j$时对$dp(n,j)$有贡献。

那$dp(n,0)$即为$\sum_{i=2}^n\ i^k$。下面考虑如何进行转移。

{\heiti 1.假设$P_j^2>n,\ P_{j-1}^2\le n\ $(临界情况)。}

那么对于$dp(n,j)$,$2\sim n$中不存在一个合数,其最小质因子$>P_j$,即$dp(n,j)$的贡献全部来自于$[2,n]$中的质数。

对于$dp(n,j-1)$,$2\sim n$中也不存在一个合数,其最小质因子$>P_{j-1}$,即$dp(n,j-1)$的贡献也全部来自于$[2,n]$中的质数。

即$dp(n,j) = dp(n,j-1)$,对于更大的$j'$,可知贡献还是不变,即$dp(n,j') = dp(n,j)$。

所以我们可以得到一个转移式:
$$
dp(n,j) = dp(n,j-1),\quad when\ P_j^2>n
$$

{\heiti 2.下面考虑$P_j^2\le n$时的情况。}

这个时候$dp(n,j-1)$有非质数的贡献,而$dp(n,j)$,如果考虑临界情况,即$P_{j+1}^2>n$,没有非质数的贡献。
也就是说$P_j^2\le n$时,从$j-1$到$j$的贡献变少了,即$dp(n,j) = dp(n,j-1) - X$。考虑这个$X$是什么。

显然$X$就是最小质因子是$P_j$的那些合数所造成的贡献。由于这样的合数,每个都有$P_j$作为质因子,我们将$P_j^k$提出,考虑剩下的部分,
即$dp(n,j) = dp(n,j-1) - P_j^k*X'$。

$X'$是什么呢?$X'$是最小质因子大于等于$P_j$的数(可以是质数)所造成的贡献,并且这些数要$\le \left\lfloor \frac{n}{P_j} \right\rfloor$。 
于是$X' = dp(\left\lfloor \frac{n}{P_j} \right\rfloor,j-1)-dp(P_j-1,j-1)$。

为什么呢?我们对$X'$造成贡献的数分成合数和质数。其中合数的贡献完全对应了式子中的被减数中的合数贡献(不多不少刚刚好);而对于质数,我们需要计数的是大于等于$P_j$的那些,
而被减数中计算了$[2,\ \left\lfloor \frac{n}{P_j} \right\rfloor]$中所有的,于是将多算的质数减去,即减去$dp(P_j-1,j-1)$。(\quad $dp(P_j-1,j-1)$的贡献全来自于
$[2,P_j-1]$中的质数\quad )

总结一下,转移式如下:
$$
dp(n,j)=
\begin{cases}
dp(n,j-1)&P_j^2 > n\\
dp(n,j-1)-P_{j}^k\ *\ [dp(\left\lfloor \frac{n}{P_j} \right\rfloor,j-1)-dp(P_j-1,j-1)]&P_j^2\le n
\end{cases}
$$

{\heiti 那我们想要求解的最终答案就是$dp(n,j)$,其中$P_j^2\le n,\ P_{j+1}^2> n$。}

由素数定理知,$j$的量级是$O(\frac{\sqrt{n}}{log\sqrt{n}})$。

\begin{note}
考虑求$dp(n,j)$单点的时间复杂度是多少?以及如何进行优化。后面我们会看到如何求$2*\sqrt{n}$个$dp$值用于$EES$。
\end{note}

\subsection{Extended Eratosthenes Sieve}
\begin{custom}{问题}
给出一个积性函数$f$,且$f(p)$为关于$p$的多项式,$p\in Prime$。
求$S(n)=\sum_{i=1}^nf(i)$,$n=10^{12}$。
\end{custom}

\begin{solution}
$\forall  \ 2\le i\le n$,我们可以将$i$分为两类:
\begin{enumerate}
	\item 第一类数:最大质因子的幂次=1,则其次大质因子$< \sqrt{n}$;
	\item 第二类数:最大质因子的幂次$>$1 ,则其最大质因子$\le \sqrt{n}$。
\end{enumerate}

\vbox{}

{\heiti $EES$算法流程如下:}
\begin{itemize}
	\item 初始化$S(n)=f(1)$ ,记$M=\left\lfloor \sqrt{n} \right\rfloor$;
	\item 枚举那些所有质因子均$\le M$的数$k$,设其最大质因子为$L$,则
	
	$S(n)+=f(k)\ *\ \sum_{L<p\le \frac n k}f(p), \quad p \ is \ prime $,此时每个$k\cdot p$都对应第一类数,且能覆盖所有第一类数;
	\item 枚举时,若$k$的最大质因子次幂$>1$,$S(n)+=f(k)$,此时$k$就是一个第二类数,且能覆盖所有第二类数。
\end{itemize}
\begin{note}
	具体实现时采用dfs。此步骤其实算是EES的第二步,第一步需要做一些预处理,即使用上面提到的$MLLMO\ Method$,具体预处理啥呢?往下看。
\end{note}

\vbox{}

{\heiti 几点说明:}
\begin{itemize}
	\item dfs时需要注意,如果对于当前枚举的基数$now$(一开始为1),有$now*p*p>n$,则不调用$now' = now*p$的$dfs$(\quad 因为$now' * newp > n,\ where\ newp>p$\quad ),同时也不继续枚举当前素数的指数,因为没有贡献了。
	\item 如果我们可以$O(1)$地求出$\sum_{L<p\le \frac n k}f(p)$,那么上面过程的时间复杂度是$\Theta(n^{1-\omega})$,但是对于$n \leq 10^{13}$
	这样的数据范围还是很快的。感兴趣的可以阅读2018年集训队论文《一些特殊的数论函数求和问题》---朱震霆。
	\item 设$g(i)=\sum_{2\le p\le i}f(p),\ p\in Prime$,现在问题只剩下了如何$O(1)$求$\sum_{L<p\le \frac n k}f(p)=g(\lfloor \frac n k\rfloor)-g(L)$。
	由于$\lfloor \frac n k\rfloor$只有$O(\sqrt{n})$种,$L\le \sqrt{n}$也只有$O(\sqrt{n})$种,因此我们只需要计算$g$的$O(\sqrt{n})$项。
	在题设里提到了$f(p)$是一个关于$p$的多项式,即$f(p)=\sum a_ip^{k_i}$,我们对于每个$i$,假设$f(p)=p^{k_i}$,最后乘上系数$a_i$再累加就可以得到$ans$。
	
	{\heiti 因此现在的问题就是求$\sum_{2\le p\le i}p^{k},\ p\in Prime$,其中$i$分别取$2,3,...,M,\frac{N}{M},\frac{N}{M-1},...,\frac{N}{2},\frac{N}{1}$。
	$M=\left\lfloor \sqrt{n} \right\rfloor$,除法是下取整。}那这个问题就可以用上面说到的$MLLMO\ Method$。算法流程如下。
\end{itemize}
	
\vbox{}
	
{\heiti 使用$MLLMO\ Method$求解$O(\sqrt{N})$个$g$值:}

\begin{itemize}
	\item 记$2,3,...,M,\frac{N}{M},\frac{N}{M-1},...,\frac{N}{2},\frac{N}{1}$为集合$NS$。对于集合$NS$中的每个数$i$,初始化$Map[i] = \sum_{2\le j\le i}j^k$。
	当$k$较小时,对于每个$Map[i]$可以由公式$O(1)$求出。{\color{red} // 相当于$dp(i,0)$}
	\item for p = 2,3,5,7,...(不超过$M$的所有素数,升序):{\color{red} // 相当于枚举dp中的$j$为$1,2,3...$}
	
	\quad \quad for 集合$NS$中的每个元素$i$(降序):
	
	\quad \quad \quad \quad if $p*p\le i$:
	
	\quad \quad \quad \quad \quad \quad $Map[i]-=(Map[i/p] - Map[p-1])\ *\ p^k$
	
	\quad \quad \quad \quad else break
	
	\quad \quad end for
	
	end for
\end{itemize}

\begin{note}
	这里滚动掉了一维$dp$数组,因此第二维要降序枚举$i$。
	时间复杂度为$O(\frac {n^{\frac 3 4}}{ log n})$。
	
	这一步预处理可看成$EES$的第一步。
\end{note}
	
\vbox{}
	
\end{solution}

至此,$EES$介绍完了,总时间复杂度为$O(\frac {n^{\frac 3 4}}{ log n})+\Theta(n^{1-\omega})$。

下面来看一些例题。

\vbox{}

\begin{example}
	输入一个数$N,\ 2\le N\le 10^{10}$,求$S(n)\ mod\ 1e9+7$。其中$S(n)=\sum_{i=1}^{n}\phi(i)$。(51nod-1239, 欧拉函数前缀和)
\end{example}

\lstinputlisting[language=C++, style=codestyle2]{code05/51nod1239.cpp}





\vbox{}





\begin{example}
	输入两个数$a,b,\ 2\le a\le b\le 10^{10}$,求$S(b)-S(a-1)\ mod\ 1e9+7$,即区间值。其中$S(n)=\sum_{i=1}^{n}\mu(i)$。(51nod-1244, 莫比乌斯函数前缀和)
\end{example}
 
\lstinputlisting[language=C++, style=codestyle2]{code05/51nod1244.cpp}






\vbox{}





\begin{example}
	定义$\sigma(n)=n$的因子数  ,求$\sum_{i=1}^n\sigma(i^k)  \ mod  \ 2^{64}$。输入两个数$n,k\ ;\ n,k \le 10^{10}$。(\href{https://www.spoj.com/problems/DIVCNTK/}{SPOJ DIVCNTK})      
\end{example}
  
\lstinputlisting[language=C++, style=codestyle2]{code05/DIVCNTK.cpp}






\vbox{}




\begin{example}
定义$f(n)=n$的最小质因子,求$\sum_{i=1}^nf(i)   \ mod \ 2^{64}$,$1\le N\le 1234567891011$。   
(\href{https://www.spoj.com/problems/APS2/}{SPOJ APS2})
\end{example}

\lstinputlisting[language=C++, style=codestyle2]{code05/APS2.cpp}

\begin{note}
	同样,也可以求最大质因子,\href{https://projecteuler.net/problem=642}{projecteuler-642}。说明了EES对非积性函数的可行性。
\end{note} 
 
 
 
 
\vbox{}




\begin{example}
输入一个数$N$,输出第$N$个素数。$1\le N\le 10^9$。 时间限制:$2.6s$。
(\href{https://www.spoj.com/problems/NTHPRIME/en/}{SPOJ NTHPRIME})
\end{example}

\begin{solution}
	使用二分答案 + MLLMO Method,时间复杂度为$O(\frac {n^{\frac 3 4}}{ log n} * log n)$。由于常数比较大,可以固定二分次数上限,剩下的部分用区间素数筛法或者$miller-rabin$。
	
	下面的代码使用的是$miller-rabin$。{\heiti 实际上区间素数筛速度更快,$1e7$长度的可筛区间,只要不到1$s$的时间,而$miller-rabin$常数比较大。}
\end{solution}

\lstinputlisting[language=C++, style=codestyle2]{code05/nthprime.cpp}

\begin{note}
考虑这题有没有更快的算法,时间主要花在二分时多次$MLLMO$上!如果能从数学上找到一个估计函数,先大致估计一下结果,然后再微调,就只要做一次$MLLMO$。
下面尝试寻找这样的估计函数进行优化。
\end{note}

\vbox{}

论文\cite{1002.0442}中给出了这样一组上下界:

$$
\begin{aligned} \pi(x) & \geqslant \frac{x}{\ln x}\left(1+\frac{1}{\ln x}+\frac{2}{\ln ^{2} x}\right) \quad \text { for } x \geqslant 88783 \\ \pi(x) & \leqslant \frac{x}{\ln x}\left(1+\frac{1}{\ln x}+\frac{2.334}{\ln ^{2} x}\right) \quad \text { for } x \geqslant 2953652287 \end{aligned}
$$
其中$\pi(x)$表示$1\sim x$中素数的个数。这样对于输入所给的$\pi(x)$,我们分别用两个估计函数解出$x$(看做等号),即可得到一个估计的范围$[L,R]$。然后只需要对$L$这一点做一次$MLLMO$,对区间$[L,R]$做大区间素数筛,
最后遍历一遍统计即可。至于如何解这两个方程,二分即可。

对于单组询问$n,\ n\le 10^9$,本机测试不超过$150ms$。
\lstinputlisting[language=C++, style=codestyle2]{code05/nthprime2.cpp}




\vbox{}






\begin{example}
	print the last prime P such that $\sum_{i=2}^Pi,\ where\ i \ is\ prime = S$。 时间限制:$3s$。
	
	The lonely line of input contains an integer S.$\ 0 < P <= 10^{12}$.
	
	(\href{https://www.spoj.com/problems/SUMPRIM2/}{SPOJ Sum of primes (reverse mode)})
\end{example}
\begin{solution}
	直接想法,和上一题一开始一样,直接二分答案,使用$log$次$MLLMO$,但由于$MLLMO$是求素数和而不是素数个数,所以多了乘法以及要使用$\_\_int128$,导致常数更大了。
	这种思路肯定是过不了。
	
	考虑寻找两个函数对答案上下界进行估计,记$S(x)$为$1\sim x$中所有素数的和。论文\cite{Axler2014On}中给出了这样一组上下界:
	$$
	\begin{aligned} S(x) & > \frac{x^2}{2logx} + \frac{x^2}{4log^2x} + \frac{x^2}{4log^3x} + \frac{1.2x^2}{8log^4x} \quad \text { for } x \geqslant 905238547 \\ S(x) & < \frac{x^2}{2logx} + \frac{x^2}{4log^2x} + \frac{x^2}{4log^3x} + \frac{5.3x^2}{8log^4x} \quad \text { for } x \geqslant 110118925 \end{aligned}
	$$
	然后就是和上一题一样的流程。对于$P=10^{12}$的极限情况,估计的范围长度最坏在$2.5*1e7$以内,可以区间素数筛。
	
	\lstinputlisting[language=C++, style=codestyle2]{code05/sum-of-primes.cpp}
\end{solution}

\vbox{}

\begin{note}
	总结一下,$2s$内可解决的问题:
		\begin{table}[!htbp]
			\centering
			\caption{关于素数问题的总结 \label{tab:summaryforprime}}
			\begin{spacing}{1}
				\begin{tabular}{|c|c|c|}
					\toprule[1pt]
					问题 & 范围 & 方法 \\
					\midrule[1.5pt]
					\tabincell{c}{求$1\sim n$中素数个数 \\ (给一个素数$p$,判断是第几个素数)} &   $n= 1e12$ &  $MLLMO$ \\
					\midrule[1pt]
					求$1\sim n$中素数的和 &  $n=1e12$ & $MLLMO$  \\
					\midrule[1pt]
					求第$n$个素数  &  $n=1e9 $ &  $\pi(x)$估计 + $MLLMO$ \\
					\midrule[1pt]
					\tabincell{c}{给一个$S$,保证是素数的前缀和, \\ 求组成$S$的最后一个素数$p$}  &   $p=1e12$ &  $S(x)$估计 + $MLLMO$ \\
					\bottomrule[1pt]
				\end{tabular}%
			\end{spacing}
		\end{table}%
\end{note}

\vbox{}


\begin{example}
	For $n=p_1^{k_1}p_2^{k_2}...p_m^{k_m}$, $\ $define$f(n)=k_1 + k_2 + ... + k_m$,$\ $please calculate $\sum_{i=1}^n f(i!)\ \%998244353$。
	输入一个$n$, $1\le n \le 10^{10}$。
	(\href{https://nanti.jisuanke.com/t/41390}{2019徐州网络赛H})
\end{example}
\begin{solution}
	显然$f(pq) = f(p) + f(q)$。
\end{solution}





\vbox{}




\begin{example}
	$f(n,k)$ is the number of way to select $k$ numbers $a_i$,$\ a_i>1$ and $\prod_{i=1}^{k} a_{i} = n$。
	solve $\sum_{i = 1} ^ {n} f(i,k)$ after mod 1e9+7.$\ $Note that if n=6, 6=2*3 and 6=3*2 are different way.
	
	Input one line contains two integers $n,k\ (1\le n\le 2^{30},1\le k\le 30)$.
	
	(\href{http://acm.hdu.edu.cn/showproblem.php?pid=6537}{2019ccpc湖南全国邀请赛 F.Neko and function})
\end{example}

\begin{solution}
	考虑$f$函数如何计算,发现不是积性函数而且不好计算。
	考虑另外一个函数$g$,和$f$的区别只是限制条件$a_i>1$变为$a_i\ge 1$,这个时候考虑$g(n,k)$如何求解以及$f$和$g$之间的关系。
	\begin{itemize}
		\item 考虑$f$和$g$之间的关系,显然$g$的情况是不小于$f$的,考虑多的部分,其实就是有$a_i=1$的情况。自然想到枚举哪个位置放1,剩下的用$g$继续表示,即$f(n,k) = g(n,k) - C_k^1*g(n,k-1)$。 但发现其实多减去了,因为后面那个部分统计显然有重复,重复的就是$C_k^2*g(n,k-2) - C_k^3*g(n,k-3)+....$   
		
		也就是要容斥一下,即$f(n,k) = \sum_{i=0}^k(-1)^i*C_k^i*g(n,k-i)$。  
		
		由于$k$只有30,这个时候祈祷$g$是积性函数,ees就结束了。确实是。
		\item 考虑函数$g(n,k)$,将$n$质因子分解,可知{\heiti 不同质因子贡献是满足积性的}。而对于$g(p^e,k)$,对应于$e$个球,$k$个盒子,球没有区别,位置(盒子)有区别的放球模型(可以有空盒)。即$g(p^e,k) = C_{e+k-1}^{k-1}$  ($k-1$个插板 ,$e+k-1$个位置)。
	\end{itemize}
	
	最多30次ees即可。注意特判$n=1$的情况。
\end{solution}

\lstinputlisting[language=C++, style=codestyle2]{code05/Neko-and-function.cpp}



\section{rng58-clj等式}



\begin{problemset}
	\item xx
\end{problemset}