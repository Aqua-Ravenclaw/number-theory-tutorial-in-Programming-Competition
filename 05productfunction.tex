\chapter{积性函数}

\begin{introduction}
	\item 因子次幂和函数
	\item 欧拉函数
	\item 莫比乌斯函数
	\item 莫比乌斯反演
	\item 积性函数与狄利克雷卷积
	\item 积性函数前缀和
	\item 杜教筛
	\item 拓展埃氏筛
	% \item rng58-clj等式
\end{introduction}


\section{因子次幂和函数}
\begin{definition}{因子次幂和函数}{label}
定义$t$次幂$\sigma$函数$\sigma_t(n)=\sum_{d|n}d^t$

两个特殊情况:

\begin{enumerate}
\item $t=0$, $\sigma_0(n)$是因子个数函数,计算方法是质因子分解后指数+1相乘;
\item $t=1$, $\sigma_1(n)$是因子和函数,计算方法是:
$$
\prod \frac{p_{i}^{n_{i}+1}-1}{p_{i}-1}
$$
\end{enumerate}

他们都是积性函数。
\end{definition}

使用欧拉筛预处理前maxn项因子和函数的值(因子个数函数类似)。时间复杂度$O(n)$。

help[maxn]存每个数最小质因子的指数次方,如$12$存$2^2$,$18$存$2^1$,$32$存$2^5$。

\lstinputlisting[language=C++, style=codestyle2]{code05/Sum-function-of-factors.cpp}

\begin{note}
对因子和函数$g(n)$再求前缀和可得到函数$f(n)=\sum_{i=1}^{n}\left \lfloor \frac{n}{i} \right \rfloor*i$。
也即$g(n)=f(n)-f(n-1)$。类似的,对因子个数函数再求前缀和可得到函数$f(n)=\sum_{i=1}^{n}\left \lfloor \frac{n}{i} \right \rfloor$。

{\heiti 因此可以在$O(\sqrt{n})$的时间内求得这类函数(因子和、因子个数)的前缀和,这类分块的思想是杜教筛的基础。}
\end{note}

\section{欧拉函数}
\begin{definition}{欧拉函数}{label}
欧拉函数$\varphi(n)$,表示小于等于$n$的数中与$n$互质的数的数目。

即$\varphi(n)=\sum_{i=1}^{n}{[(n,i)=1]\cdot 1}$。
\end{definition}

相关性质:
\begin{itemize}
\item $\varphi(1)=1$, $\varphi(x)$是偶数$where \ x>2$;
\item 若$n$是素数$p$的$k$次幂,$\varphi(n)=p^k-p^{k-1}=(p-1)p^{k-1}$;
\item 若$m,n$互质,$\varphi(mn)=\varphi(m)\varphi(n)$;
\item 当$n$为奇数时,有$\varphi(n)=\varphi(2n)$;{\heiti (可以理解为此时$2$是$2n$的最小质因子)}
\item $\varphi(n)=n*(1-\frac{1}{p_1})*(1-\frac{1}{p_2})*...*(1-\frac{1}{p_k}),\quad p_1,p_2,...,p_k$是$n$的质因子;
\item $n=\sum_{d|n}{\varphi(d)}$ ,质因数分解乘起来即可证明;
\item $\sum_{i=1}^{n}{[(n,i)=1]\cdot i}=\frac{n\cdot\varphi(n)+[n=1]}{2}$;
\item $\varphi(n)=\sum_{d|n}{\mu(d)\frac{n}{d}}$;{\heiti (由$n=\sum_{d|n}{\varphi(d)}$ 莫比乌斯反演即得)}
\end{itemize}

\vbox{}

几个式子总结:
\begin{itemize}
\item $\sum_{d|n}\varphi(d)=n$;{\color{red} $\varphi * id = I$}
\item $\sum_{d|n}\varphi(\frac{n}{d})\cdot d\ =\sum gcd(i, n)(1<=i <=n)\ =\ $$[0,n)$中任选两个数$a,b$,且$a*b$是$n$的倍数的方案数;    {\color{red} $\varphi * I$}
\item $\sum_{d|n}\sum_{w|d}\varphi(\frac{d}{w})\cdot w=n $乘以 ($n$的因子数目)。  {\color{red} $\varphi * I * id = I * I$}
\end{itemize}

\vbox{}

{\heiti 求解欧拉函数单值}

若时间卡的比较紧,先筛选下素数。

\lstinputlisting[language=C++, style=codestyle2]{code05/euler.cpp}

\vbox{}

{\heiti 欧拉函数线性筛}

$p$为$N$的素因子,若$p^2 | N$,则$\varphi(N)=\varphi(\frac{N}{p})*p$ ;否则$\varphi(N)=\varphi(\frac{N}{p})*(p-1)$。
做素数筛时顺带处理即可。

\lstinputlisting[language=C++, style=codestyle2]{code05/euler-linear.cpp}

\vbox{}

{\heiti 下面来看一些例题。}

\vbox{}

\begin{example}
求$\sum gcd(i, N)(1<=i <=N)$。\quad [HYSBZ - 2705 ]
\end{example}

\begin{solution}
枚举哪些数$d$作为$i$和$N$的$gcd$,则$d=gcd(i,N)$,即$1=gcd(i/d,N/d)$。记答案为$f(N)$,有$f(N)=\sum_{d|N}\varphi(\frac{N}{d})*d$为答案。

同时$f(n)$还表示$[0,n)$中任选两个数$a,b$,且$a\cdot b$是$n$的倍数的方案数。(下面例2)

由于是积性函数,素数次幂欧拉值可以$O(1)$得到,只剩下质因子分解的时间。总时间复杂度$O(\sqrt{n})$。
\end{solution}

\vbox{}

\begin{example}
定义$f(n)=$选两个$[0,n)$的整数$a,b$,且$ab$不是$n$的倍数的方案数。求$g(n)=\sum_{d|n}f(d)$。
数据组数$1\le T\le 20000,\ 1\le n \le 10^9$。  \quad [HDU - 5528 ] icpc2015长春B
\end{example}

\begin{solution}
我们考虑$f(n)$怎么计算,$f(n)$等于$n^2$减去$ab$是$n$的倍数的方案数。

令$h(n)$表示$ab$是$n$的倍数的方案数,则$h(n)=\sum_{d|n}\varphi(\frac{n}{d})\cdot d$。      
如何理解呢?我们去枚举$gcd(a,n)$的值,可知这个值会等于$n$的某个因子,因此我们枚举$n$的因子,则当$gcd(a,n)=d$时,即$gcd(\frac{a}{d},\frac{n}{d})=1 $,$a$有$\varphi(\frac{N}{d})$种取值,由于a只有$N$的因子d,则$b$要包含$N/d$这个因子,那么有多少$b$满足呢?即$N/N/d=d$个 。因此$h(n)=\sum_{d|n}\varphi(\frac{n}{d})\cdot d$。 

所以所求为$g(n)=\sum_{d|n}f(d)=\sum_{d|n}d^2-\sum_{d|n}h(d)$。  

对于第二项,质因子分解后,则$\sum_{d|p^k}h(d)=\sum_{d|p^k}\sum_{d'|d}\varphi(\frac{d}{d'})d'$。 

对于$p^k$的因子只有$k+1$个,即$log$级别,因此总共只有质因子分解的时间。

{\heiti 总时间复杂度$O(T*(\frac{\sqrt{n}}{log(n)}+log^2n))$}, 注意这里质因子分解时用素数去筛(先预处理出素数) ,否则$TLE$。

{\heiti 本题还有一种思路:}  我们知道$\sum_{d|n}\varphi(d)=n$,其卷积形式为$\varphi * id=id*\varphi =I$,  其中$I$为恒等函数,$id$为单位函数。本题我们主要求解的是$\sum_{d|n}h(n)=\sum_{d|n}\sum_{w|d}\varphi(\frac{d}{w})\cdot w=
\sum_{d|n}(\varphi * I)_{(d)}=\varphi* I *id=(\varphi* id )*I=I*I= \sum_{d|n}I(d)\cdot 
I(\frac{n}{d})=\sum_{d|n}d\cdot \frac{n}{d}=\sum_{d|n}n=n\cdot \sum_{d|n}1 =n$ 乘以 ($n$的因子数目)。     

\begin{note}
这里卷积操作$*$指的是狄利克雷卷积,后面会介绍。
\end{note}

\end{solution}

\vbox{}

\begin{example}
给出$T$组$N,M$,求出$\sum_{i=1}^{N}\sum_{j=1}^Mgcd(i,j)$的值。 $N,M \ 1e6,\quad T\ 1e3$。
\end{example}

\begin{solution}
考虑$n=\sum_{d|n}\varphi(d)$,则$\sum_{i=1}^{N}\sum_{j=1}^Mgcd(i,j)=\sum_{i=1}^{N}\sum_{j=1}^M  \sum_{d|gcd(i,j)}\varphi(d)$。 

可以对于每个$d$,去考虑$i,j$的贡献,则有$\sum_{i=1}^{N}\sum_{j=1}^M  \sum_{d|gcd(i,j)}\varphi(d)=\sum_d\varphi(d)*(\sum_{1\le i\le N \ and \ d|i}  \sum_{1\le j\le M \ and \ d|j}1)  =\sum_d\varphi(d)* \left \lfloor \frac{N}{d}\right \rfloor  \left \lfloor \frac{M}{d}\right \rfloor$。

如果直接暴力上式子,时间复杂度是$O(min(N,M))$。   

由于出现了下取整符号,考虑进一步优化:分块块。

$\left \lfloor \frac{N}{d} \right \rfloor$ 和$\left \lfloor \frac{M}{d}\right \rfloor$在一起分块的区间
最多只有$2\left \lfloor \sqrt{n}\right \rfloor+2\left \lfloor \sqrt{m}\right \rfloor$ 段。

{\heiti 总时间复杂度}    $O(1e6+T*(\sqrt{n}+\sqrt{m}))$    
\end{solution}

\lstinputlisting[language=C++, style=codestyle2]{code05/euler-example3.cpp}

\section{莫比乌斯函数}
莫比乌斯函数$\mu(n)$是做莫比乌斯反演的时候一个很重要的系数。

\begin{definition}{莫比乌斯函数}{label}
$$
\mu(x)=\left\{\begin{array}{c}{1, x=1} \\ {(-1)^{k}, x=p_{1} p_{2} \cdots p_{k}} \\ {0, x=o t h e r s}\end{array}\right.
$$
\end{definition}

相关性质:
\begin{itemize}
\item $[n=1]=\sum_{d|n}{\mu(d)}$ ,将$\mu (d)$看作是容斥的系数即可证明。 也可以记作$e=\mu * id$,其中$e$为元函数,即$\mu$在狄利克雷卷积的乘法中与单位函数$id$互为逆元。 
\begin{proof}
设$n^*=p_1...p_k$  可知只有当$d|n^*$时,$\mu$值有贡献,所以$n>1$时有:       
$$
\sum_{d|n}\mu(d)=\sum_{d|n^*}\mu(d)=\sum_{i=0}^kC_k^i(-1)^i=(1-1)^k=0
$$
\end{proof}

\item $\varphi(n)=\sum_{d|n}{\mu(d)\frac{n}{d}}$, 也写作 $\sum_{d|n}{\frac{\mu(d)}{d}}=\frac{\phi(n)}{n}$, {\heiti 由$n=\sum_{d|n}{\varphi(d)}$ 莫比乌斯反演即得}。
\begin{note}
后面会介绍莫比乌斯反演。
\end{note}
\end{itemize}

\vbox{}

{\heiti 求莫比乌斯函数单值}

$O(\sqrt{n}/log\sqrt{n})$
\lstinputlisting[language=C++, style=codestyle2]{code05/mobiwusi.cpp}

\vbox{}

{\heiti 莫比乌斯函数线性筛}
$O(n)$
\lstinputlisting[language=C++, style=codestyle2]{code05/mobi-linear.cpp}

\vbox{}

\begin{example}
求$1<=x,y<=N$且$Gcd(x,y)$为素数的数对$(x,y)$有多少对。$N 1e7$。\quad (bzoj2818)
\end{example}

\begin{solution}
\begin{enumerate}
\item {\color{red}  方法一。}

枚举小于$N$的素数,由于$x,y$可以交换,不妨设$x<y$,则求$gcd(x,y)=p$等价于$gcd(x/p,y/p)=1$。枚举
素数,对于一个素数,再枚举$y=p,2p,...,kp$ ,$x$的方案数即求一个欧拉函数的前缀和。于是先预处理出欧拉函数前缀。

{\heiti 时间复杂度:}预处理$O(1e7)$,每个询问是$O(N/logN)$。 

\item {\color{red}  方法二。}

枚举素数$p$,令$n=\lfloor\frac{N}{p}\rfloor$,则求
$$
\sum_{i=1}^n\sum_{j=1}^{n}[gcd(i,j)=1]
$$
(其实到这一步直接用欧拉函数就行了,即法1,下面用莫比乌斯函数)即求
$$
\sum_{i=1}^n\sum_{j=1}^{n}\sum_{d|gcd(i,j)}\mu(d)
$$
即
$$
\sum_{d=1}^n\mu(d)*\sum_{i=1}^{\lfloor\frac{n}{d}\rfloor}\sum_{j=1}^{\lfloor\frac{n}{d}\rfloor}1
$$
即
$$
\sum_{d=1}^n\mu(d)\lfloor\frac{n}{d}\rfloor\lfloor\frac{n}{d}\rfloor
$$
由于我们要枚举素数$p$,即这里的$n=N/p$  ,如果我们纯暴力计算上面和式(先筛出$\mu$),时间复杂度
是$O(\sum_{i=1}^N\frac{N}{i},\ where \ i \ is \ prime)=O(NlogN/logN)\approx O(N)$。  

这样复杂度比方法一是要差的。因为方法一是$O(N/logN)$。
考虑可不可以进一步优化?  

对于一个$n$,上面和式中,$\lfloor\frac{n}{d}\rfloor$只有$\sqrt{n}$种取值,即一段$d$对应的$\lfloor\frac{n}{d}\rfloor$值是相同的,
因此只要有连续一段的$\mu$的和值(预处理前缀和),即可$O(\sqrt n)$求解上面和式。

{\heiti 因此时间复杂度是(对于每个询问): $O(\sum_{i=1}^n\sqrt{\frac{N}{i}},\ where \ i \ is \ prime)=O(N/logN)$。

时间复杂度同方法一。

本题是一道经典的题目,使用了常见的$\varphi,\mu$两种函数求解。}

$O(N/logN)$不是很优秀,考虑能不能进一步优化。

\item {\color{red}  方法三。 \quad (bzoj2820)} 

即现在要求$\sum_p\sum_{d=1}^{\frac{N}{p}}\mu(d)\lfloor\frac{\lfloor \frac{N}{p} \rfloor }{d}\rfloor\lfloor\frac{ \lfloor \frac{N}{p} \rfloor}{d}\rfloor=\sum_p\sum_{d=1}^{\frac{N}{p}}\mu(d)\lfloor \frac{N}{pd}\rfloor\lfloor\frac{N}{pd}\rfloor$。      

考虑枚举$D=pd$ ,则原式等于$\sum_D\lfloor \frac{N}{D} \rfloor \lfloor \frac{N}{D} \rfloor\sum_{p|D}\mu(\frac{D}{p})\ ,\ where\ p \ is \ prime$。   

令$g(n)=\sum_{p|n}\mu(\frac{n}{p}) \ ,\ where \ p \ is \ prime $ ,可以预处理出其前缀和,因此可以分块块。  

{\heiti 时间复杂度:}   $O(1e7)+T*O(\sqrt{N})$。
\end{enumerate}
\end{solution}

\lstinputlisting[language=C++, style=codestyle2]{code05/mobi-bzoj2820.cpp}

\vbox{}

\begin{example}
给出$T$组$N,M$,求出$\sum_{i=1}^{N}\sum_{j=1}^Mlcm(i,j)$的值。     $N,M\le 1e7,\ T\le 1e3$。     [HYSBZ - 2154 ]
\end{example}

\begin{solution}
\label{exa:lcm}
如果求$gcd$,直接用欧拉函数经典公式换掉分块即可。

而对于$lcm$,原式等价于$\sum_{i=1}^{N}\sum_{j=1}^M\frac{ij}{gcd(i,j)}$。 

我们考虑枚举$d=gcd(i,j)$,则有原式$=\sum_{d}\sum_{i=1}^{\lfloor\frac{N}{d}\rfloor}\sum_{j=1}^{\lfloor\frac{M}{d}\rfloor}\frac{di\cdot dj}{d}\ ,\ where \ i,j \ is \ co-prime=\sum_{d}d \sum_{i=1}^{\lfloor\frac{N}{d}\rfloor}\sum_{j=1}^{\lfloor\frac{M}{d}\rfloor}[gcd(i,j)==1]ij$。 

为了方便,我们定义函数$f(n,m)=\sum_{i=1}^n\sum_{j=1}^m[gcd(i,j)==1]ij$, 这个式子很经典,经常出题。  

出现了熟悉的$[==1]$,我们用$\mu$换掉,则得$f(n,m)=\sum_{i=1}^{n}\sum_{j=1}^{m}\sum_{d'|gcd(i,j)}\mu(d')ij$。 

如果从正面考虑(枚举$i,j$),复杂度不行,考虑枚举$d'$,则$f(n,m)=\sum_{d'}\mu(d')\cdot (d'+2d'+...+\lfloor \frac{n}{d'}\rfloor d')\cdot (d'+2d'+...+\lfloor \frac{m}{d'}\rfloor d')=\sum_{d'} \mu(d')\cdot (\sum_{i=1}^{\lfloor \frac{n}{d'}\rfloor}id'\sum_{j=1}^{\lfloor \frac{m}{d'}\rfloor}jd') =\frac{1}{4}\sum_{d'}\mu(d')d'^{2}\cdot \lfloor \frac{n}{d'}\rfloor(\lfloor \frac{n}{d'}\rfloor+1)\cdot \lfloor \frac{m}{d'}\rfloor(\lfloor \frac{m}{d'}\rfloor+1)$。 

有了$f(n,m)$的式子,原式$=\frac{1}{4}\sum_dd\cdot \sum_{d'}\mu(d')d'^{2}\cdot \lfloor \frac{\lfloor\frac{N}{d} \rfloor }{d'}\rfloor(\lfloor \frac{\lfloor\frac{N}{d} \rfloor }{d'}\rfloor+1)\cdot \lfloor \frac{\lfloor\frac{M}{d} \rfloor }{d'}\rfloor(\lfloor \frac{\lfloor\frac{M}{d} \rfloor }{d'}\rfloor+1) =\frac{1}{4}\sum_dd\cdot \sum_{d'}\mu(d')d'^{2}\cdot \lfloor \frac{N}{dd'}\rfloor(\lfloor \frac{N }{dd'}\rfloor+1)\cdot \lfloor \frac{M}{dd'}\rfloor(\lfloor \frac{M }{dd'}\rfloor+1)$。

这个式子直接去做的话,分两次块块,时间复杂度是$O(n^{\frac{3}{4}})$。 

考虑这里$d$和$d'$的特殊性,由于$d\cdot d'\le min(N,M)$,我们可以枚举$D=d\cdot d'$的值,
则原式$=\frac{1}{4}\sum_{D}\lfloor \frac{N}{D}\rfloor(\lfloor \frac{N }{D}\rfloor+1)\cdot \lfloor \frac{M}{D}\rfloor(\lfloor \frac{M }{D}\rfloor+1)\cdot (D\cdot \sum_{d|D}\mu(d)\cdot d)$。  

设函数$g(n)=n\sum_{d|n}\mu(d)d$,可知$g(n)$是积性函数,且可以线性时间筛出。

因此只要一次分块块即可,对于连续的一块,由维护的$g(n)$的前缀和可$O(1)$得到连续的一段$g(D)$的值。

{\heiti 时间复杂度:    预处理$O(1e7)$,每个询问是$O(\sqrt{N}+\sqrt{M})$。

注:  $g(n) = n\sum_{d|n}{d\mu(d)}$   是一个积性函数,可以用欧拉筛搞出。注意$\mu$很好,当有大于2次幂时贡献为0了。}
\end{solution}

\lstinputlisting[language=C++, style=codestyle2]{code05/mobi-HYSBZ2154.cpp}



\section{莫比乌斯反演}
\begin{definition}{莫比乌斯反演的一种形式}{label}
设$F$是$f$的因子和型函数,即
$$
F(n)=\sum_{d|n}{f(d)}
$$
那么对于所有的正整数$n$,都有下面式子成立
$$
f(n)=\sum_{d|n}{\mu(d)F(\frac{n}{d})}
$$
即已知$F(n)$,我们可以求出$f(n)$。    
\end{definition}
   
{\heiti 通过欧拉函数的莫比乌斯反演进一步理解容斥与反演:}

我们知道欧拉函数有:$n=\sum_{d|n}\varphi(d)$,由莫比乌斯反演可得$\varphi(n)=\sum_{d|n}\mu(d)\cdot \frac{n}{d}$。

我们知道欧拉函数的意义是小于n的数中与n互质的数的个数,
那为啥这个$\mu$乘一乘、加一加就能表示欧拉函数了呢?

就是容斥啦,我们考虑$[1,n]$这$n$个数,我们要找的是一些数$k$,有$gcd(k,n)=1$,那显然可能还会有一些数$k$,有$gcd(k,n)>1$   ,那我们要从$n$个数中,减去$gcd>1$的那些数$k$,这些数有多少呢?

考虑将$n$质因子分解得到质因子们$p_1,p_2,...,p_r$ ,显然有$\frac{n}{p_1}$  个$p_1$的倍数$gcd>1$,同理$\frac{n}{p_2},\frac{n}{p_3},...,\frac{n}{p_r}$ ,将这些个数减去;

但会有重复,比如有一个数$p_1p_2$,在两个地方都减去了,要再加回来,同理所有的两素数都是这样,所以加上$\frac{n}{p_1p_2}$等等两个素数作为分母;

再考虑有一个数$p_1p_2p_3$,考虑一个素数时减去了3次,两个时又加上了3次,因此没有减去这样的数,所以要减去$\frac{n}{p_1p_2p_3}$等等三个素数作为分母;

这个就是{\heiti 容斥原理}。

再观察$\varphi(n)=\sum_{d|n}\mu(d)\cdot \frac{n}{d}$,由于$\mu$的定义,这里有贡献的$d$恰好取遍了一个素数、两个素数......的容斥情况,而当$d=1$时,$\mu(1)=1$,即全体个数$n$。


\section{积性函数与狄利克雷卷积}
\subsection{积性函数定义}
{\heiti 数论函数:}若$f(x)$的定义域为正整数域,值域为复数域,即$f:\mathbb{Z}^+  \rightarrow \mathbb{C}$ ,则称$f(x)$为数论函数。

{\heiti 积性函数:}若$f(x)$为数论函数,$f(1)=1$,且$gcd(m,n)=1$时$f(x)$对所有整数$m$与$n$满足乘法公式$f(mn)=f(m)f(n)$,则称$f(x)$为积性函数。

{\heiti 完全积性函数:}若$f(x)$为数论函数,且对任意正整数$m,n$,都有$f(mn)=f(m)f(n)$,则称其为完全积性函数。



\subsection{常见的积性函数}
\begin{itemize}
\item 元函数\quad $e(n)=[n=1]$,狄利克雷卷积中的乘法单元,完全积性
\item 单位函数\quad $id(n)=1$,完全积性
\item 恒等函数\quad $I(n)=n$,完全积性
\item 幂函数\quad $id^k(n)=n^k$,完全积性
\item 欧拉函数$\varphi(x)$
\item 莫比乌斯函数$\mu(x)$
\item $t$次幂$\sigma$函数 $\sigma_t(n)=\sum_{d|n}d^t$
\item 因子个数函数$\sigma_0(n)$
\item 因子和函数$\sigma_1(n)$
\item 刘维尔函数 $\lambda(n)$

\qquad 定义为将$n$分解后,$n=p_1^{k_1}p_2^{k_2}...p_r^{k_r}$,令$\lambda(n)=(-1)^{k_1+k_2+...+k_r}$。$\lambda(1)=1$  。令$\Omega(n)= k_1+k_2+...+k_r$     则$\lambda(n)=(-1)^{\Omega(n)}$ 。
\begin{note}
1.刘维尔函数的狄利克雷逆变换是莫比乌斯函数的绝对值。

2.刘维尔函数与莫比乌斯函数存在下面式子:
$$
\left\{\begin{matrix}
\sum_{d|n}\lambda(d)\mu(\frac{n}{d}) =1  ,\quad n=k^2 \\ 
\sum_{d|n}\lambda(d)\mu(\frac{n}{d}) =0,\quad  n\neq k^2 
\end{matrix}\right.
$$
\end{note}

\item 使用刘维尔函数定义新的函数$G(n)$,
$$
G(n)=\lambda(d_1)+\lambda(d_2)+...+\lambda(d_r)=\sum_{d|n}\lambda(d)
$$
$G(n)$也是积性函数。其满足下面的性质:
$$
G(n)=\sum _{{d|n}}\lambda (d)={\begin{cases}1&{\text{if }}n{\text{ is a perfect square,}}\\0&{\text{otherwise.}}\end{cases}}
$$
\end{itemize}


\subsection{不是很常见的积性函数}
\begin{itemize}
\item $g(n) = n\sum_{d|n}{d\mu(d)}$  是积性函数。

可以用欧拉筛搞出。注意$\mu$很好,当有大于2次幂时贡献为0了;{\color{red} (\ref{exa:lcm}节中的例题用到了)}
\item $f(n)=n\sum_{d|n}d\varphi(d)=-n+2*\sum_{i=1}^nLCM(n,i)$是积性函数。    

其$f(p^{k+1})=p*f(p^k)+p^{3(k+1)}-p^{3k+2}=p^3f(p^{k})-p^{k+1}(p-1)$,注意有减法,因此可能要维护一下每个数的最小质因子的指数,也可以$O(n)$预处理。
{\color{red} (\ref{exa:lcm2}节中的例题用到了)}
\end{itemize}


\subsection{“作用在因子上”}

定义函数$F(n)=\sum_{d|n}f(d)$,若$f(x)$是积性函数,则$F(x)$也是积性函数;反之,若$F(x)$为积性函数,则$f(x)$也是积性函数。

\begin{proof}
$f(x)$是积性函数,对于$gcd(m,n)=1$,要证明$F(mn)=F(m)F(n)$。设:
$d_1,d_2,...,d_r$是$n$的因数,$e_1,e_2,...,e_s$是$m$的因数。

$m,n$互质,则$mn$的因子就是上面$r$个因子和$s$个因子两两相乘得到的,共$r*s$个。

又$d_i$和$e_j$互质,则$f(d_ie_j)=f(d_i)f(e_j)$  ,则
\begin{align*}
F(mn)&=f(d_1e_1)+f(d_1e_2)+...+f(d_1e_s)+...+f(d_r)f(e_1)+...+f(d_r)f(e_s) \\& = [f(d_1)+f(d_2)+...+f(d_r)]*[f(e_1)+f(e_2)+...+f(e_s)] \\ 
&=F(n)*F(m)
\end{align*}
证毕。

反之,当$F(n)$为积性函数时,证明类似。
\end{proof}


\subsection{狄利克雷卷积}
\begin{definition}{狄利克雷卷积}{label}
设$f,g$为两个数论函数,则满足$h(n)=\sum_{d|n}f(d)g(\frac{n}{d})$的函数称为$f$与$g$的狄利克雷卷积函数。也可以理解为$h(n)=\sum_{ij=n}f(i)g(j)$ 。
\begin{itemize}
\item 两个积性函数的卷积仍为积性函数;
\item 狄利克雷卷积满足交换律和结合律,对加法满足分配律。  
\end{itemize}
\end{definition}

\begin{note}
多个函数的狄利克雷卷积类似。$\sum_{ijkl..z=n}f_1(i)f_2(j)f_3(k)...f_{*}(z)$。
\end{note}

\vbox{}

常见的卷积:
\begin{itemize}
\item $\varphi * id = I$
\item $\mu * id = e$
\end{itemize}



\subsection{一些题目}

\begin{example}
定义$f_0(n)=$满足$pq=n$且$gcd(p,q)=1$的二元组$(p,q)$个数。定义$f_{r+1}(n) = \sum_{uv=n}\frac{f_r(u)+f_r(v)}{2}$。
若干组询问,每次给出$r,n$,询问$f_r(n)$的值,对$10^9+7$取模。询问次数$1\le q\le 10^6$,$0\le r\le 10^6$,$1\le n\le 10^6$。
\href{http://codeforces.com/contest/757/problem/E}{cf-757E}
\end{example}

\begin{solution}
满足$pq=n$且$gcd(p,q)=1$的二元组个数是多少呢?

显然是$2^{w(n)}$个,其中$w(n)$是$n$的不同的素因子的数目。

那么$f_{r+1}(n)=\sum_{uv=n}\frac{f_r(u)+f_r(v)}{2}$如何计算呢?

$f_{r+1}(n)=\sum_{uv=n}\frac{f_r(u)+f_r(v)}{2}=\sum_{d|n}f_r(d)$。 

由于$r\le 1e6$,询问又有$1e6$,那我们不可能暴力搞。

我们知道$f_r(n)$是积性函数,所以问题转为求解$f_r(p^k)$。

由于$f_r(p^k)=\sum_{i=0}^{i=k}f_{r-1}(p^i)$,且对任意$p$,$k>=1$时,有$f_0(p^k)=2$,因此$f_r(p^k)$的值和$p$无关。预处理所有的$r,k$组合即可。$k$是$log$级别的。

{\heiti 时间复杂度:}  $O(rlogn+T*(\frac{\sqrt{n}}{log10^3}))$   
\end{solution}

\lstinputlisting[language=C++, style=codestyle2]{code05/cf757E.cpp}

\vbox{}

\begin{example}
$T$个$N$,依次计算$\sum_{a=1}^{N}\sum_{b=1}^NLCM(a,b)$ \quad             $N,T\le 10^6$
\end{example}

\begin{solution}
\label{exa:lcm2}
前面我们解过$\sum_{a=1}^{N}\sum_{b=1}^MLCM(a,b)$,复杂度可以是$T*O(\sqrt{N}+\sqrt{M})$。显然解这题是不够的。

考虑本题,如果直接令一个函数$f(N)=\sum_{a=1}^{N}\sum_{b=1}^NLCM(a,b)$,其是不是积性函数呢?

可惜不是,{\heiti 但这个函数$f(n)=-n+2*\sum_{i=1}^nLCM(n,i)$是积性函数。}

为啥长这样... 其实这样很美,因为$\sum_{i=1}^Nf(i)$即为本题所求。

但$f(n)$目前这个样子还是不知道怎么搞,我们化简一下。

设$gcd(n,i)=d$,则$f(n)=-n+2\sum_{i=1}^n\frac{ni}{d}=-n+2n\sum_{d|n}\sum_{i\le n \ \& \ gcd(i,n)=d}\frac{i}{d}=-n+2n\sum_{d|n}\sum_{k\le \frac{n}{d}\ \& \ gcd(k,\frac{n}{d})=1}k=-n+2n\sum_{d|n}\sum_{k\le d\  \& \ gcd(k,d)=1}k=-n+2n((\sum_{d|n\ \& \ d>1}\frac{d\varphi(d)}{2})+1)=n\sum_{d|n}d\varphi(d)$。 

即$f(n)=-n+2*\sum_{i=1}^nLCM(n,i)=n\sum_{d|n}d\cdot \varphi(d)$ 是积性函数。

所以,只要筛出这个积性函数,就可以$O(1)$回答一个询问。

如何去筛这个函数呢?考虑一个数$p^k$,若它新出现一个质因子变为$p^kq$,则函数值$f(p^kq)=f(p^k)f(q)$,若从$p^k$变为$p^{k+1}$,则$f(p^{k+1})=p*f(p^k)+p^{3(k+1)}-p^{3k+2}=p^3f(p^{k})-p^{k+1}(p-1)$   , 后面的部分是$p\varphi(p^{k+1})$,因此欧拉筛搞一搞即可 (因为有减法,而不仅仅是乘法,可能要维护一下每个数最小质因数指数幂后的数)。

{\heiti 时间复杂度:} $O(1e6+T)$

\end{solution}

\section{积性函数前缀和}
{\heiti 一些式子:}

\begin{itemize}
\item $$
\left\lceil\frac{n}{a b}\right\rceil=\left\lceil \frac{\left\lceil \frac{n}{a}\right\rceil}{b}  \right\rceil, \quad\left\lfloor\frac{n}{a b}\right\rfloor=\left\lfloor\frac{\left\lfloor\frac{n}{a}\right\rfloor}{b}\right\rfloor
$$

\item $\left\lceil\frac{n}{i}\right\rceil$只有$O(\sqrt{n})$种取值。
\end{itemize}

$\left \lceil \frac{n}{i} \right \rceil$ 为什么只有$O(\sqrt{n})$种取值呢?若$i<\sqrt{n}$ 则由于$i$只有$\sqrt{n}$种取值,所以$\left \lceil \frac{n}{i} \right \rceil$只有$\sqrt{n}$种取值;若$i>\sqrt{n}$  ,则$\left \lceil \frac{n}{i} \right \rceil<\sqrt{n}$  ,所以 $\left \lceil \frac{n}{i} \right \rceil$最多也只有$\sqrt{n}$种取值。

\vbox{}

{\heiti 一些常用复杂度计算:}

\begin{itemize}
\item $\sum_{i=1}^{n}\frac{n}{i}=O(nlogn)$
\item $$\sum_{i=1}^{n^x}\sqrt{\frac{n}{i}}=O(n^{(\frac{1}{2}+\frac{1}{2}x)})$$

例如$\sum_{i=1}^{\sqrt{n}}\sqrt{\frac{n}{i}}=O(n^{\frac{3}{4}}$)     ,   $\sum_{i=1}^{n}\sqrt{\frac{n}{i}}=O(n)$   ,$\sum_{i=1}^{n^{\frac{1}{3}}}\sqrt{\frac{n}{i}}=O(n^{\frac{2}{3}})$。
\end{itemize}

\vbox{}

{\heiti 前缀和:}

给出一个积性函数$f(x)$,求$\sum_{i=1}^nf(i)$。

通常,$f(p^k)$可以比较容易的由$f(p^{k-1})$等值递推出来,其他项可以直接由积性函数的性质由$f(x)=f(d)*f(\frac{x}{d})$得到。因此很多积性函数都可以在欧拉筛的过程中顺便递推出前$n$项的值,时间复杂度$O(n)$。

\subsection{一些题目}

\begin{example}
设$\sigma(n)$为因子和函数,现在要求解$\sum_{i=1}^n\sigma(i)$。 
\end{example}


\begin{solution}
一种简单的思考方法,考虑函数$f(n)=\sum_{i=1}^ni*\lfloor \frac{n}{i}\rfloor$。  

发现$f(n)-f(n-1)=\sum_{i=1}^ni*(\lfloor \frac{n}{i}\rfloor-\lfloor \frac{n-1}{i}\rfloor)=\sigma(n)$。  

因此$f(n)$为我们所求,分块块即可$O(\sqrt{n})$。
\end{solution}

\begin{note}
因此我们也可以线性筛出函数$f(1...n)=\sum_{i=1}^ni*\lfloor \frac{n}{i}\rfloor$。 
\end{note}

\vbox{}

\begin{example}
线性筛出函数$f(n)=\sum_{i=1}^n\left \lceil \frac{n}{i}  \right \rceil$ ,即求$f(1),f(2),...,f(n)$。
\end{example}


\begin{solution}
注意下取整和上取整的不同,下取整的差分减一对应下标减一的因子个数函数,即$f(n)-f(n-1)=\tau(n-1)+1$。

因此先预处理出因子个数函数,再移位+1,最后求前缀和。
\end{solution}

\lstinputlisting[language=C++, style=codestyle2]{code05/Factor-number-function-prefix.cpp}

\vbox{}

\begin{example}
求$ans =\sum^n_{i=1}\sum^i_{j=1} (n\  mod (i \times j))$     \quad     $1\le n \le 10^{11}$   \quad 
\href{https://www.oj.swust.edu.cn/contest/problem/1253-G}{18四川省赛G}
\end{example}


\begin{solution}
{\heiti \color{red} 解法一}

首先写作$\sum^n_{i=1}\sum^i_{j=1} (n-\left \lfloor \frac{n}{ij} \right \rfloor*ij)=\sum^n_{i=1}\sum^i_{j=1} n-\sum^n_{i=1}\sum^i_{j=1}\left \lfloor \frac{n}{ij} \right \rfloor*ij $      

对于$\sum^n_{i=1}\sum^i_{j=1} n$不用说了,对于$\sum^n_{i=1}\sum^i_{j=1}\left \lfloor \frac{n}{ij} \right \rfloor*ij =\sum^n_{i=1}i*\sum^i_{j=1}\left \lfloor\frac{\left \lfloor \frac{n}{i} \right \rfloor} {j}\right \rfloor*j$ 

对于第二维,所求是j$\in [1,i]$,我们这里求$[1,n]$,由对称性可知$[1,i]$的两倍减去$[i=j]$的情况即为所求。

所以现在目标是$\sum^n_{i=1}i*\sum^n_{j=1}\left \lfloor\frac{\left \lfloor \frac{n}{i} \right \rfloor} {j}\right \rfloor*j$。 

由于$\left \lfloor \frac{n}{i} \right \rfloor$ 只有$O(\sqrt{n})$种取值,我们枚举$\left \lfloor \frac{n}{i} \right \rfloor$的值,对应一段$i$区间,可知对于这一段$i$,第二维值是相同的。

而对于第二维的计算相同,都是分块思想。所以写一个solve函数用于求解$f(n)=\sum_{i=1}^{n}\left \lfloor \frac{n}{i} \right \rfloor*i$ 即可。 

{\heiti 时间复杂度:  $O(\sum_{i=1}^{\sqrt{n}}\sqrt{\frac{n}{i}})=O(n^{\frac{3}{4}})$  (只有$O(\sqrt{n})$种取值)         }

这种方法会TLE,      1e11大概要跑37s。

{\heiti \color{red} 解法二  (常见套路)}   

定义$g(n)=f(n)-f(n-1)=\sum_{i=1}^{n}i* (\left \lfloor \frac{n}{i}\right \rfloor-\left \lfloor \frac{n-1}{i} \right \rfloor)=\sum_{d|n}d$ 。   

对于$g(n)$,即一个数的因子和,我们可以$O(n)$求解出其前$n$项,对于本题$n \ 1e11$ ,我们处理出$g$的前$n^{\frac{2}{3}}$项 ,然后求个前缀和就得到了$f(n)$ ,这一部分的时间复杂度是$O(n^\frac{2}{3})$  。  

所以对于$\sum^n_{i=1}i*\sum^n_{j=1}\left \lfloor\frac{\left \lfloor \frac{n}{i} \right \rfloor} {j}\right \rfloor*j$,当$\left \lfloor \frac{n}{i}\right \rfloor<n^\frac{2}{3}$时,可以$O(1)$得到第二层结果。当$\left \lfloor \frac{n}{i}\right \rfloor>n^{\frac{2}{3}}$时,使用原先的方法求解,这一部分的时间复杂度是$O(\sum_{i=1}^{n^\frac{1}{3}}\sqrt{\frac{n}{i}})=O(n^\frac{2}{3})$。

{\heiti 时间复杂度: $O(n^\frac{2}{3})$,       1e11大概要跑5s}
\end{solution}

\lstinputlisting[language=C++, style=codestyle2]{code05/18sichuanG.cpp}



\section{杜教筛}
设$S(n)=\sum_{i=1}^nf(i)$,下面将求解$S(n)$的过程一般化:

任意数论函数$g$,设$h=f*g$({\color{red} 狄利克雷卷积}),有$\sum_{i=1}^nh(i)=\sum_{i=1}^ng(i)S(\lfloor \frac{n}{i}\rfloor)$   (改变计数方式易得)

从右式分出一个$i=1$,移项可得$g(1)S(n)=\sum_{i=1}^nh(i)-\sum_{i=2}^ng(i)S(\lfloor \frac{n}{i}\rfloor)$

{\heiti 如果我们可以$O(\sqrt{n})$计算$\sum_{i=1}^nh(i)$,$O(1)$计算$g$的前缀和,就可以快速把问题递归为同类子问题},时间复杂度为$O(\sum_{i=1}^{\sqrt{n}}\sqrt{\frac{n}{i}})=O(n^{\frac{3}{4}})$       (具体计算这里略去)

如果$f$有一些比较好的性质,比如是积性函数,我们可以用欧拉筛求出前$n^{\frac{2}{3}}$项,更后面的项再递归,时间复杂度为$O(n^{\frac{2}{3}})$。 

\subsection{一些要注意的地方}
\begin{itemize}
\item 会卡map;
\item 一个记忆化的技巧是,由于我们要求解的始终是$S(n除去一些数)$,因此我们可以开一个数组,比如$p[]$,$p[i]$的地方存储$S(\frac{n}{i})$的值,显然由于我们预处理了$i<n^{\frac{2}{3}}$的情况,则$p$数组的大小只有$n^{\frac{1}{3}}$大小;
\item 如果出现$\sum_{i=1}^N\sum_{d|i}$的形式要求解,直接对后面杜教筛可能不太好搞,因为找不到合适的狄利克雷卷积或者不好找。我们可以尝试调换贡献的枚举(这里枚举$i$是$d$的多少倍),变换为$\sum_{i=1}^N\sum_{d=1}^{\lfloor \frac N i \rfloor}f(d)$的形式,对后面一个求$\sum f$的问题做杜教筛,而前面不影响复杂度(分$i<n^{\frac 1 3}$和$>$考虑)。如:

\qquad 要求函数$f(i) = i \sum_{d | i} d \varphi(d)$的前缀和,

$$
\begin{array}{l}{\sum_{i=1}^{N} f(i)=\sum_{i=1}^{N} i \sum_{d | i} d \varphi(d)} \\ {=\sum_{d=1}^{N} d \varphi(d) \sum_{k=1}^{\frac{N}{d}} k d} \\ {=\sum_{d=1}^{N} d^{2} \varphi(d) \sum_{k=1}^{\frac{N}{d}} k} \\ {=\sum_{k=1}^{N} k \sum_{d=1}^{\left\lfloor\frac{N}{k}\right\rfloor} \phi(d) d^{2}} \\ {=\sum_{i=1}^{N} i \sum_{d=1}^{\left\lfloor\frac{N}{i}\right\rfloor} \phi(d) d^{2} \quad \left( replace\ symbol\ k\ with\ i\right)}  \end{array}
$$ 
如果$f(x)=\varphi(x)x^2\ ,\ S(n)=\sum_{i=1}^nf(i)$  可以做杜教筛,则上面的式子也行,不影响时间复杂度 $O(n^{\frac 2 3})$。

以及下面的第二、三个例子\ref{exa:dujiao1}也是类似的;
\item $f(x)=\varphi(x)x^k$都是可以搞的,即$h(x)=x^{k+1}\ ,\ g(x)=x^k$。
\end{itemize}

\subsection{一些题目}
\begin{example}
求$\sum_{i=1}^n\varphi(i)$和$\sum_{i=1}^n\mu(i),\quad n\le 1e9$  \quad 多组
\end{example}
\begin{solution}
\begin{itemize}
\item 考虑有没有函数$g$,这个函数易于计算前缀和,且和$\varphi$狄利克雷卷积后的函数也较易于计算前缀和。
对于欧拉函数,这是显然存在的,即$g(n)=id(n)\ ,\ h(n)=n\ ,\ I = \varphi * id$。 

设$S(n)=\sum_{i=1}^n\varphi(i),\ g(n)=1$。则$g(1)S(n)=\sum_{i=1}^nh(i)-\sum_{i=2}^ng(i)S(\lfloor \frac{n}{i}\rfloor)$。 

带入$g,h$的值后,有$S(n)=\sum_{i=1}^ni-\sum_{i=2}^nS(\lfloor \frac{n}{i}\rfloor)$。

用欧拉筛求出欧拉函数的前$n^{\frac{2}{3}}$,求前缀和得到$S$的前$n^{\frac{2}{3}}$项。$\lfloor \frac{n}{i}\rfloor$ 较大时递归计算,总时间复杂度$O(n^\frac{2}{3})$。

\item 对于 $梅滕斯函数(Mertens)=\sum_{i=1}^n\mu(i)$,同样的,我们寻找函数$g$,有$h=\mu*g$,其中$g$可以$O(1)$计算前缀和。

我们知道$\mu$的一个性质是$\sum_{d|n}\mu(d)=[n==1]$,  它的卷积形式是$e=\mu * id$。

因此$g(1)S(n)=\sum_{i=1}^nh(i)-\sum_{i=2}^ng(i)S(\lfloor \frac{n}{i}\rfloor)$,化简后得$S(n)=1-\sum_{i=2}^nS(\lfloor \frac{n}{i}\rfloor)$  。
\end{itemize}
\end{solution}
\lstinputlisting[language=C++, style=codestyle2]{code05/varphi-mu.cpp}

\vbox{}


\begin{example}
求\quad (51nod1237)
$$
\sum_{i=1}^{n}\sum_{j=1}^{n}gcd(i,j)
$$

\end{example}
\begin{solution}
\label{exa:dujiao1}
\begin{align*}
ans=2\sum_{i=1}^n\sum_{j=1}^igcd(i,j)-\frac {n(n+1)}{ 2}=2\sum_{i=1}^n\sum_{d|i}\varphi(\frac i d)d-\frac {n(n+1)}{ 2} \\ =2\sum_{d=1}^nd\sum_{k=1}^{\lfloor \frac n d \rfloor}\varphi(k)  -\frac {n(n+1)}{ 2}
\end{align*}
这里对第一维分块不影响复杂度,只要对第二维做杜教筛即可,$I=\varphi * id$。
\end{solution}


\vbox{}


\begin{example}
求\quad {51nod 1238}
$$
ans=\sum_{i=1}^n\sum_{j=1}^n LCM(i,j)    \quad  n \le 10^{10}
$$
\end{example}
\begin{solution}

\begin{align*}
ans=\sum_{i=1}^n\sum_{j=1}^n [i,j]\\
=2\sum_{i=1}^n\sum_{j=1}^i [i,j]-\frac{n(n+1)}2\\
Let\ s(n)=\sum_{i=1}^n\sum_{j=1}^i [i,j]\ ,\ f(n)=\sum_{i=1}^n [i,n]\\
\end{align*}

{\color{red}
\begin{align*}
f(n)=\sum_{i=1}^n [i,n]\\
=\sum_{i=1}^n\frac{in}{(i,n)}\\
=n\sum_{i=1}^n\frac i{(i,n)}\\
=n\sum_{d|n}\sum_{i=1}^n[(i,n)=d]\frac i d\\
=n\sum_{d|n}\sum_{i=1}^{\frac n d}[(i,\frac n d)=1]i\\
=n\sum_{d|n}\sum_{i=1}^{d}[(i,d)=1]i\quad (due\ to\ the\ symmetry\ of\ the\ factor)\\
=n\sum_{d|n}\frac{\phi(d)d+[d=1]}2\\
=n\frac{1+\sum_{d|n}\phi(d)d}2
\end{align*}
}

{\color{blue}
\begin{align*}
s(n)=\sum_{i=1}^n f(i)\\
=\frac{\sum_{i=1}^n i(1+\sum_{d|i}\phi(d)d)}2\\
=\frac{\sum_{i=1}^n i+\sum_{i=1}^n i\sum_{d|i}\phi(d)d}2\\
=\frac{\frac{n(n+1)}2+\sum_{i=1}^n i\sum_{d|i}\phi(d)d}2\\
=\frac{\frac{n(n+1)}2+\sum_{d=1}^n\phi(d)d\sum_{d|i}i}2\\
=\frac{\frac{n(n+1)}2+\sum_{d=1}^n\phi(d)d^2\sum_{i=1}^{\lfloor\frac n d\rfloor}i}2\\
=\frac{\frac{n(n+1)}2+\sum_{i=1}^n i\sum_{d=1}^{\lfloor\frac n i\rfloor}\phi(d)d^2}2\\
\end{align*}
}

{\color{green}
\begin{align*}
ans=2s(n)-\frac{n(n+1)}2\\
=\sum_{i=1}^n i\sum_{d=1}^{\lfloor\frac n i\rfloor}\phi(d)d^2\\
\end{align*}
}

可以看到杜教筛的形式,这里对第一维分块不影响复杂度,只要对第二维做杜教筛即可。
\begin{align*}
Here's\ how\ to\ make\ a\ Dujiao\ sieve\ for\ \phi(d)d^2\\
Let\ f(d)=\phi(d)d^2\ ,\ S(n)=\sum_{d=1}^nf(d)\\
n=\sum_{d|n}\phi(d)\\
n^3=\sum_{d|n}\phi(d)n^2\\
=\sum_{d|n}\phi(d)d^2(\frac n d)^2 \quad Now\ you\ can\ see\ that\ it's\ n^2\ and\ n^3\\ 
=\sum_{d|n}h(d)(\frac n d)^2\\
So\ \sum_{i=1}^n i^3=\sum_{i=1}^n\sum_{d|i}f(d)(\frac i d)^2\\
=\sum_{d=1}^n f(d)\sum_{d|i}(\frac i d)^2\\
=\sum_{d=1}^n f(d)\sum_{i=1}^{\lfloor\frac n d \rfloor}i^2\\
=\sum_{i=1}^n i^2\sum_{d=1}^{\lfloor\frac n i\rfloor}f(d)\\
=\sum_{i=1}^n i^2 S(\lfloor\frac n i\rfloor)\\
S(n)=\sum_{i=1}^n i^3-\sum_{i=2}^ni^2 S(\lfloor\frac n i\rfloor)
\end{align*}
\end{solution}




\section{拓展埃氏筛}
拓展埃氏筛(Extended Eratosthenes Sieve)似乎最早由min25引入竞赛圈(因此也常被称为min25筛),可以在{\heiti 低于线性的时间}内求得积性函数(一些非积性函数也可以)的前缀和。
由于其方法简单且灵活性高,近年来经常在算法竞赛中出现,下面就来介绍一下EES。

首先要介绍的是在EES中要使用到的一个前置算法,叫做The Meissel, Lehmer, Lagarias, Miller, Odlyzko Method{\heiti (MLLMO Method)},论文\cite{Deleglise1996Computing}中
对其进行了系统介绍,有兴趣的可以继续阅读。

\subsection{MLLMO Method}
MLLMO Method要解决的是这样一个问题:
求解
$$
\sum_{i=2}^n[i\in Prime]F(i)
$$
其中$F(x) = x^k$,即求解
$$
\sum_{i=2}^n[i\in Prime]\ i^k
$$
设$P_j$表示第$j$个质数,$P_1=2,\ P_2=3\ ...$,特殊地$P_0$可以认为是$0$。

MLLMO Method用动态规划的思想解决这个问题:

设
$$
dp(n,j)=\sum_{i=2}^ni^k\ [\ i\in Prime\ or \ min(p)>P_j,\ where\ p|i\ and\ p\in Prime\ ]
$$
也就是说$i$是质数,或者合数$i$的最小质因子大于$P_j$时对$dp(n,j)$有贡献。

那$dp(n,0)$即为$\sum_{i=2}^n\ i^k$。下面考虑如何进行转移。

{\heiti 1.假设$P_j^2>n,\ P_{j-1}^2\le n\ $(临界情况)。}

那么对于$dp(n,j)$,$2\sim n$中不存在一个合数,其最小质因子$>P_j$,即$dp(n,j)$的贡献全部来自于$[2,n]$中的质数。

对于$dp(n,j-1)$,$2\sim n$中也不存在一个合数,其最小质因子$>P_{j-1}$,即$dp(n,j-1)$的贡献也全部来自于$[2,n]$中的质数。

即$dp(n,j) = dp(n,j-1)$,对于更大的$j'$,可知贡献还是不变,即$dp(n,j') = dp(n,j)$。

所以我们可以得到一个转移式:
$$
dp(n,j) = dp(n,j-1),\quad when\ P_j^2>n
$$

{\heiti 2.下面考虑$P_j^2\le n$时的情况。}

这个时候$dp(n,j-1)$有非质数的贡献,而$dp(n,j)$,如果考虑临界情况,即$P_{j+1}^2>n$,没有非质数的贡献。
也就是说$P_j^2\le n$时,从$j-1$到$j$的贡献变少了,即$dp(n,j) = dp(n,j-1) - X$。考虑这个$X$是什么。

显然$X$就是最小质因子是$P_j$的那些合数所造成的贡献。由于这样的合数,每个都有$P_j$作为质因子,我们将$P_j^k$提出,考虑剩下的部分,
即$dp(n,j) = dp(n,j-1) - P_j^k*X'$。

$X'$是什么呢?$X'$是最小质因子大于等于$P_j$的数(可以是质数)所造成的贡献,并且这些数要$\le \left\lfloor \frac{n}{P_j} \right\rfloor$。 
于是$X' = dp(\left\lfloor \frac{n}{P_j} \right\rfloor,j-1)-dp(P_j-1,j-1)$。

为什么呢?我们对$X'$造成贡献的数分成合数和质数。其中合数的贡献完全对应了式子中的被减数中的合数贡献(不多不少刚刚好);而对于质数,我们需要计数的是大于等于$P_j$的那些,
而被减数中计算了$[2,\ \left\lfloor \frac{n}{P_j} \right\rfloor]$中所有的,于是将多算的质数减去,即减去$dp(P_j-1,j-1)$。(\quad $dp(P_j-1,j-1)$的贡献全来自于
$[2,P_j-1]$中的质数\quad )

总结一下,转移式如下:
$$
dp(n,j)=
\begin{cases}
dp(n,j-1)&P_j^2 > n\\
dp(n,j-1)-P_{j}^k\ *\ [dp(\left\lfloor \frac{n}{P_j} \right\rfloor,j-1)-dp(P_j-1,j-1)]&P_j^2\le n
\end{cases}
$$

{\heiti 那我们想要求解的最终答案就是$dp(n,j)$,其中$P_j^2\le n,\ P_{j+1}^2> n$。}

由素数定理知,$j$的量级是$O(\frac{\sqrt{n}}{log\sqrt{n}})$。

\begin{note}
考虑求$dp(n,j)$单点的时间复杂度是多少?以及如何进行优化。后面我们会看到如何求$2*\sqrt{n}$个$dp$值用于$EES$。
\end{note}

\subsection{Extended Eratosthenes Sieve}
\begin{custom}{问题}
给出一个积性函数$f$,且$f(p)$为关于$p$的多项式,$p\in Prime$。
求$S(n)=\sum_{i=1}^nf(i)$,$n=10^{12}$。
\end{custom}

\begin{solution}
$\forall  \ 2\le i\le n$,我们可以将$i$分为两类:
\begin{enumerate}
	\item 第一类数:最大质因子的幂次=1,则其次大质因子$< \sqrt{n}$;
	\item 第二类数:最大质因子的幂次$>$1 ,则其最大质因子$\le \sqrt{n}$。
\end{enumerate}

\vbox{}

{\heiti $EES$算法流程如下:}
\begin{itemize}
	\item 初始化$S(n)=f(1)$ ,记$M=\left\lfloor \sqrt{n} \right\rfloor$;
	\item 枚举那些所有质因子均$\le M$的数$k$,设其最大质因子为$L$,则
	
	$S(n)+=f(k)\ *\ \sum_{L<p\le \frac n k}f(p), \quad p \ is \ prime $,此时每个$k\cdot p$都对应第一类数,且能覆盖所有第一类数;
	\item 枚举时,若$k$的最大质因子次幂$>1$,$S(n)+=f(k)$,此时$k$就是一个第二类数,且能覆盖所有第二类数。
\end{itemize}
\begin{note}
	具体实现时采用dfs。此步骤其实算是EES的第二步,第一步需要做一些预处理,即使用上面提到的$MLLMO\ Method$,具体预处理啥呢?往下看。
\end{note}

\vbox{}

{\heiti 几点说明:}
\begin{itemize}
	\item dfs时需要注意,如果对于当前枚举的基数$now$(一开始为1),有$now*p*p>n$,则不调用$now' = now*p$的$dfs$(\quad 因为$now' * newp > n,\ where\ newp>p$\quad ),同时也不继续枚举当前素数的指数,因为没有贡献了。
	\item 如果我们可以$O(1)$地求出$\sum_{L<p\le \frac n k}f(p)$,那么上面过程的时间复杂度是$\Theta(n^{1-\omega})$,但是对于$n \leq 10^{13}$
	这样的数据范围还是很快的。感兴趣的可以阅读2018年集训队论文《一些特殊的数论函数求和问题》---朱震霆。
	\item 设$g(i)=\sum_{2\le p\le i}f(p),\ p\in Prime$,现在问题只剩下了如何$O(1)$求$\sum_{L<p\le \frac n k}f(p)=g(\lfloor \frac n k\rfloor)-g(L)$。
	由于$\lfloor \frac n k\rfloor$只有$O(\sqrt{n})$种,$L\le \sqrt{n}$也只有$O(\sqrt{n})$种,因此我们只需要计算$g$的$O(\sqrt{n})$项。
	在题设里提到了$f(p)$是一个关于$p$的多项式,即$f(p)=\sum a_ip^{k_i}$,我们对于每个$i$,假设$f(p)=p^{k_i}$,最后乘上系数$a_i$再累加就可以得到$ans$。
	
	{\heiti 因此现在的问题就是求$\sum_{2\le p\le i}p^{k},\ p\in Prime$,其中$i$分别取$2,3,...,M,\frac{N}{M},\frac{N}{M-1},...,\frac{N}{2},\frac{N}{1}$。
	$M=\left\lfloor \sqrt{n} \right\rfloor$,除法是下取整。}那这个问题就可以用上面说到的$MLLMO\ Method$。算法流程如下。
\end{itemize}
	
\vbox{}
	
{\heiti 使用$MLLMO\ Method$求解$O(\sqrt{N})$个$g$值:}

\begin{itemize}
	\item 记$2,3,...,M,\frac{N}{M},\frac{N}{M-1},...,\frac{N}{2},\frac{N}{1}$为集合$NS$。对于集合$NS$中的每个数$i$,初始化$Map[i] = \sum_{2\le j\le i}j^k$。
	当$k$较小时,对于每个$Map[i]$可以由公式$O(1)$求出。{\color{red} // 相当于$dp(i,0)$}
	\item for p = 2,3,5,7,...(不超过$M$的所有素数,升序):{\color{red} // 相当于枚举dp中的$j$为$1,2,3...$}
	
	\quad \quad for 集合$NS$中的每个元素$i$(降序):
	
	\quad \quad \quad \quad if $p*p\le i$:
	
	\quad \quad \quad \quad \quad \quad $Map[i]-=(Map[i/p] - Map[p-1])\ *\ p^k$
	
	\quad \quad \quad \quad else break
	
	\quad \quad end for
	
	end for
\end{itemize}

\begin{note}
	这里滚动掉了一维$dp$数组,因此第二维要降序枚举$i$。
	时间复杂度为$O(\frac {n^{\frac 3 4}}{ log n})$。
	
	这一步预处理可看成$EES$的第一步。
\end{note}
	
\vbox{}
	
\end{solution}

至此,$EES$介绍完了,总时间复杂度为$O(\frac {n^{\frac 3 4}}{ log n})+\Theta(n^{1-\omega})$。

下面来看一些例题。

\vbox{}

\begin{example}
	输入一个数$N,\ 2\le N\le 10^{10}$,求$S(n)\ mod\ 1e9+7$。其中$S(n)=\sum_{i=1}^{n}\phi(i)$。
	\href{https://www.51nod.com/Challenge/Problem.html#problemId=1239}{(51nod-1239, 欧拉函数前缀和)}
\end{example}

\lstinputlisting[language=C++, style=codestyle2]{code05/51nod1239.cpp}





\vbox{}





\begin{example}
	输入两个数$a,b,\ 2\le a\le b\le 10^{10}$,求$S(b)-S(a-1)\ mod\ 1e9+7$,即区间值。其中$S(n)=\sum_{i=1}^{n}\mu(i)$。
	\href{https://www.51nod.com/Challenge/Problem.html#problemId=1244}{(51nod-1244, 莫比乌斯函数前缀和)}
\end{example}
 
\lstinputlisting[language=C++, style=codestyle2]{code05/51nod1244.cpp}






\vbox{}





\begin{example}
	定义$\sigma(n)=n$的因子数  ,求$\sum_{i=1}^n\sigma(i^k)  \ mod  \ 2^{64}$。输入两个数$n,k\ ;\ n,k \le 10^{10}$。(\href{https://www.spoj.com/problems/DIVCNTK/}{SPOJ DIVCNTK})      
\end{example}
  
\lstinputlisting[language=C++, style=codestyle2]{code05/DIVCNTK.cpp}






\vbox{}




\begin{example}
定义$f(n)=n$的最小质因子,求$\sum_{i=1}^nf(i)   \ mod \ 2^{64}$,$1\le N\le 1234567891011$。   
(\href{https://www.spoj.com/problems/APS2/}{SPOJ APS2})
\end{example}

\lstinputlisting[language=C++, style=codestyle2]{code05/APS2.cpp}

\begin{note}
	同样,也可以求最大质因子,\href{https://projecteuler.net/problem=642}{projecteuler-642}。说明了EES对非积性函数的可行性。
\end{note} 
 
 
 
 
\vbox{}




\begin{example}
输入一个数$N$,输出第$N$个素数。$1\le N\le 10^9$。 时间限制:$2.6s$。
(\href{https://www.spoj.com/problems/NTHPRIME/en/}{SPOJ NTHPRIME})
\end{example}

\begin{solution}
	使用二分答案 + MLLMO Method,时间复杂度为$O(\frac {n^{\frac 3 4}}{ log n} * log n)$。由于常数比较大,可以固定二分次数上限,剩下的部分用区间素数筛法或者$miller-rabin$。
	
	下面的代码使用的是$miller-rabin$。{\heiti 实际上区间素数筛速度更快,$1e7$长度的可筛区间,只要不到1$s$的时间,而$miller-rabin$常数比较大。}
\end{solution}

\lstinputlisting[language=C++, style=codestyle2]{code05/nthprime.cpp}

\begin{note}
考虑这题有没有更快的算法,时间主要花在二分时多次$MLLMO$上!如果能从数学上找到一个估计函数,先大致估计一下结果,然后再微调,就只要做一次$MLLMO$。
下面尝试寻找这样的估计函数进行优化。
\end{note}

\vbox{}

论文\cite{1002.0442}中给出了这样一组上下界:

$$
\begin{aligned} \pi(x) & \geqslant \frac{x}{\ln x}\left(1+\frac{1}{\ln x}+\frac{2}{\ln ^{2} x}\right) \quad \text { for } x \geqslant 88783 \\ \pi(x) & \leqslant \frac{x}{\ln x}\left(1+\frac{1}{\ln x}+\frac{2.334}{\ln ^{2} x}\right) \quad \text { for } x \geqslant 2953652287 \end{aligned}
$$
其中$\pi(x)$表示$1\sim x$中素数的个数。这样对于输入所给的$\pi(x)$,我们分别用两个估计函数解出$x$(看做等号),即可得到一个估计的范围$[L,R]$。然后只需要对$L$这一点做一次$MLLMO$,对区间$[L,R]$做大区间素数筛,
最后遍历一遍统计即可。至于如何解这两个方程,二分即可。

对于单组询问$n,\ n\le 10^9$,本机测试不超过$150ms$。
\lstinputlisting[language=C++, style=codestyle2]{code05/nthprime2.cpp}




\vbox{}






\begin{example}
	print the last prime P such that $\sum_{i=2}^Pi,\ where\ i \ is\ prime = S$。 时间限制:$3s$。
	
	The lonely line of input contains an integer S.$\ 0 < P <= 10^{12}$.
	
	(\href{https://www.spoj.com/problems/SUMPRIM2/}{SPOJ Sum of primes (reverse mode)})
\end{example}
\begin{solution}
	直接想法,和上一题一开始一样,直接二分答案,使用$log$次$MLLMO$,但由于$MLLMO$是求素数和而不是素数个数,所以多了乘法以及要使用$\_\_int128$,导致常数更大了。
	这种思路肯定是过不了。
	
	考虑寻找两个函数对答案上下界进行估计,记$S(x)$为$1\sim x$中所有素数的和。论文\cite{Axler2014On}中给出了这样一组上下界:
	$$
	\begin{aligned} S(x) & > \frac{x^2}{2logx} + \frac{x^2}{4log^2x} + \frac{x^2}{4log^3x} + \frac{1.2x^2}{8log^4x} \quad \text { for } x \geqslant 905238547 \\ S(x) & < \frac{x^2}{2logx} + \frac{x^2}{4log^2x} + \frac{x^2}{4log^3x} + \frac{5.3x^2}{8log^4x} \quad \text { for } x \geqslant 110118925 \end{aligned}
	$$
	然后就是和上一题一样的流程。对于$P=10^{12}$的极限情况,估计的范围长度最坏在$2.5*1e7$以内,可以区间素数筛。
	
\end{solution}

\lstinputlisting[language=C++, style=codestyle2]{code05/sum-of-primes.cpp}

\vbox{}

\begin{note}
	总结一下,$2s$内可解决的问题:
		\begin{table}[!htbp]
			\centering
			\caption{关于素数问题的总结 \label{tab:summaryforprime}}
			\begin{spacing}{1}
				\begin{tabular}{|c|c|c|}
					\toprule[1pt]
					问题 & 范围 & 方法 \\
					\midrule[1.5pt]
					\tabincell{c}{求$1\sim n$中素数个数 \\ (给一个素数$p$,判断是第几个素数)} &   $n= 1e12$ &  $MLLMO$ \\
					\midrule[1pt]
					求$1\sim n$中素数的和 &  $n=1e12$ & $MLLMO$  \\
					\midrule[1pt]
					求第$n$个素数  &  $n=1e9 $ &  $\pi(x)$估计 + $MLLMO$ \\
					\midrule[1pt]
					\tabincell{c}{给一个$S$,保证是素数的前缀和, \\ 求组成$S$的最后一个素数$p$}  &   $p=1e12$ &  $S(x)$估计 + $MLLMO$ \\
					\bottomrule[1pt]
				\end{tabular}%
			\end{spacing}
		\end{table}%
\end{note}

\vbox{}

\begin{example}
	For $n=p_1^{k_1}p_2^{k_2}...p_m^{k_m}$, $\ $define$f(n)=k_1 + k_2 + ... + k_m$,$\ $please calculate $\sum_{i=1}^n f(i!)\ \%998244353$。
	输入一个$n$, $1\le n \le 10^{10}$。
	(\href{https://nanti.jisuanke.com/t/41390}{2019徐州网络赛H})
\end{example}

\begin{solution}
	注意函数$f(x)$虽然不满足积性,但是也有很好的性质:$f(pq) = f(p) + f(q)$。这样所求的式子可以化简为$\sum_{i=1}^n g(i)$,其中$g(i)=(n-i+1)*f(i)$。
	$n$的量级是$1e10$,考虑如何$EES$。
	
	dfs时参数要有$f$(即当前数$k$的$f(k)$值),$k$(当前的数)。具体记贡献如下:
	\begin{itemize}
		\item 对于最大质因子指数为1的贡献:$\sum_{p}\left[n-k p_{i}+1\right][f(k)+f(p)]=[f(k)+1] *\left[\sum(n+1) * 1-k * \sum\left(p_{i}\right)\right]$。也就是
		说我们需要$MLLMO$预处理素数个数前缀和以及素数前缀和。
		\item 对于最大质因子指数>1的数(比如数$i$)的贡献:直接就用$g(i)=f(i)*(n-i+1)$计算。其中$i$和$f(i)$边乘边维护下即可(分别乘$p$和加1)。
	\end{itemize}
\end{solution}

\lstinputlisting[language=C++, style=codestyle2]{code05/function.cpp}



\vbox{}



\begin{example}
	$f(n,k)$ is the number of way to select $k$ numbers $a_i$,$\ a_i>1$ and $\prod_{i=1}^{k} a_{i} = n$。
	solve $\sum_{i = 1} ^ {n} f(i,k)$ after mod 1e9+7.$\ $Note that if n=6, 6=2*3 and 6=3*2 are different way.
	
	Input one line contains two integers $n,k\ (1\le n\le 2^{30},1\le k\le 30)$.
	
	(\href{http://acm.hdu.edu.cn/showproblem.php?pid=6537}{2019ccpc湖南全国邀请赛 F.Neko and function})
\end{example}

\begin{solution}
	考虑$f$函数如何计算,发现不是积性函数而且不好计算。
	考虑另外一个函数$g$,和$f$的区别只是限制条件$a_i>1$变为$a_i\ge 1$,这个时候考虑$g(n,k)$如何求解以及$f$和$g$之间的关系。
	\begin{itemize}
		\item 考虑$f$和$g$之间的关系,显然$g$的情况是不小于$f$的,考虑多的部分,其实就是有$a_i=1$的情况。自然想到枚举哪个位置放1,剩下的用$g$继续表示,即$f(n,k) = g(n,k) - C_k^1*g(n,k-1)$。 但发现其实多减去了,因为后面那个部分统计显然有重复,重复的就是$C_k^2*g(n,k-2) - C_k^3*g(n,k-3)+....$   
		
		也就是要容斥一下,即$f(n,k) = \sum_{i=0}^k(-1)^i*C_k^i*g(n,k-i)$。  
		
		由于$k$只有30,这个时候祈祷$g$是积性函数,ees就结束了。确实是。
		\item 考虑函数$g(n,k)$,将$n$质因子分解,可知{\heiti 不同质因子贡献是满足积性的}。而对于$g(p^e,k)$,对应于$e$个球,$k$个盒子,球没有区别,位置(盒子)有区别的放球模型(可以有空盒)。即$g(p^e,k) = C_{e+k-1}^{k-1}$  ($k-1$个插板 ,$e+k-1$个位置)。
	\end{itemize}
	
	最多30次ees即可。注意特判$n=1$的情况。
\end{solution}

\lstinputlisting[language=C++, style=codestyle2]{code05/Neko-and-function.cpp}

\vbox{}

\begin{example}
	题意转化后:
	函数$f$是积性函数,且$f(p^k) = (p\%4==1) \ ?\ 3k+1 : 1$。求$\sum_{i=1}^n f(i)$,其中$n\le 10^9$。
	(\href{https://ac.nowcoder.com/acm/contest/887/K}{2019牛客暑期多校训练营第七场 K.Function})
\end{example}

\begin{solution}
可以看到根据$p\%4$的值的不同,计算方法不同,这时候怎么办呢?对于大于$2$的素数$p$,$p$模4为1或3。如果我们能分别预处理它们的前缀,那么问题就得到了解决。
首先我们引入一个函数$\chi(n)$如下:

\begin{align*}
\chi(n) = \left\{\begin{matrix}
1&  \quad  n\%4==1\\
-1  & \quad  n\%4==3 \\
0 &\quad   n\%4==0\ or\ 2
\end{matrix}\right. \quad\quad n>0
\end{align*}

$\chi(n)$是一个积性函数,所以我们在$MLLMO$的时候可以乘上这个函数,这样我们就可以预处理出$1\sim n$中$\%4=1$的素数减去$\%4=3$的素数的个数(那些前缀们)。由于我们可以
方便地预处理出$1\sim n$中$\%4=1$的素数加上$\%4=3$的素数的个数(那些前缀们,也就是所有素数,就像最常做的那样),所以就能推出$1\sim n$中模4余1的素数个数了。
\end{solution}


\lstinputlisting[language=C++, style=codestyle2]{code05/nowcode-Function.cpp}


\begin{note}
	考虑有没有其他方法,比如直接从$MLLMO$的转移式考虑。
\end{note}

% \section{rng58-clj等式}


\vbox{}

\vbox{}

\begin{problemset}
	\item HDU6588\quad 2019ICPC网络赛\quad 经典gcd求和
	\item HDU6707\quad 2019CCPC网络赛\quad 杜教筛
\end{problemset}