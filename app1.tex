\chapter{常用的表}

本附录包括了计量经济学中用到的一些基本数学,我们扼要论述了求和算子的各种性质,研究了线性和某些非线性方程的性质,并复习了比例和百分数。我们还介绍了一些在应用计量经济学中常见的特殊函数,包括二次函数和自然对数,前 4 节只要求基本的代数技巧,第 5 节则对微分学进行了简要回顾;虽然要理解本书的大部分内容,微积分并非必需,但在一些章末附录和第 3 篇某些高深专题中,我们还是用到了微积分。

\section{求和算子与描述统计量}

\textbf{求和算子} 是用以表达多个数求和运算的一个缩略符号,它在统计学和计量经济学分析中扮演着重要作用。如果 $\{x_i: i=1, 2, \ldots, n\}$ 表示 $n$ 个数的一个序列,那么我们就把这 $n$ 个数的和写为:

\begin{equation}
\sum_{i=1}^n x_i \equiv x_1 + x_2 +\cdots + x_n
\end{equation}

附录:梅森素数表!!!
卡米歇尔数