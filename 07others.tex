\chapter{其他}
\begin{introduction}[本章内容提要]
	\item xxx
\end{introduction}

\section{勾股数组}

\section{循环节问题}
\subsection{模线性方程循环节}

\subsection{斐波那契循环节}

\subsection{分数小数的循环节}
\begin{custom}{问题}
给定一个分数$\frac{p}{q}$,将其转化为小数。
\end{custom}

\begin{solution}
考虑一个分数,有以下三种情况:
\begin{itemize}
	\item 整数
	\item 有限小数
	\item 无限循环小数
\end{itemize}

先来考虑下无限循环小数,我们先将$p$和$q$公共的因子除去,然后利用循环这个性质,可知存在$i,j$满足,$p*10^i \equiv p*10^j \ (mod \ q), \ where\ j>i\ge0$。
{\heiti 这里$i$代表循环出现前有多少位数,$j-i$表示循环节长度。}
化简后得$p*10^i*(10^{j-i}-1) \equiv 0\   (mod \ q)$。也就是说,$q\ |\ p*10^i*(10^{j-i}-1)$。由于$10^{j-i}-1$中一定不存在因子$2$和$5$,所以要想是$q$的倍数,
$2$和$5$的贡献均来自$10^i$。于是我们记录$q$中$2$的次幂数为$num2$,$5$的次幂数为$num5$,则$i = max(num2,num5)$。记$q$除去所有$2$和$5$因子后,为$m$,则$10^{j-i}\equiv 1 \ (mod\ m)$。
然后我们解同余方程$10^x\equiv 1\ (mod\ m),\ where\ x>0$,$j$就等于$x+i$。至于这个方程,$x$一定是$\varphi(m)$的因子,枚举因子判定即可。

如果$j$无解,表示该小数为有限小数,$i$也就表示了小数点后共有几位;如果$j$有解,含义如上。
\end{solution}

\href{https://www.luogu.org/problem/P1530}{luogu P1530 分数化小数}

时间复杂度:$\sqrt{q}*log(q)$ \quad (确定i,j)

输出格式: 按照下面规则,如果结果长度大于76,每行输出76个字符。
\begin{itemize}
	\item 2/2 = 1.0
	\item 3/8 = 0.375
	\item 45/56 = 0.803(571428)
\end{itemize}

\lstinputlisting[language=C++, style=codestyle2]{code07/fraction2decimal.cpp}

\begin{note}
	由于上面这个题目要求输出时每行只要$76$个字符,我在写代码时一开始没有注意,因此直接使用的cout。所以代码中我使用了下面的方法:
	\lstinputlisting[language=C++, style=codestyle2]{code07/redirect-cout.cpp}
	
	先将cout重定向到stringstream,然后从其中取出string,最后将cout还原到stdout。
\end{note}

\section{DFS Similar}
\subsection{反欧拉函数}

\subsection{因子个数最多的数}

\subsection{Counting Sequences I(18上海网络赛D)}

\section{平方数之和}

\subsection{将数分解为平方之和}


\subsection{将素数分解为平方之和}

\section{FFT 与 NTT}

\subsection{FFT}

\subsection{NTT}

\section{多项式}

\subsection{拉格朗日插值法}

\subsection{多项式求逆}

\subsection{多项式取模}

\subsection{多项式开方}

\subsection{多项式多点求值}

\subsection{多项式多点插值}



\begin{problemset}
	\item xx
\end{problemset}


\nocite{*} 

\bibliography{reference}