\chapter{其他}
\begin{introduction}[本章内容提要]
	\item 勾股数组
	\item 圆上整点与高斯素数
	\item 模线性方程循环节
	\item 分数转小数
	\item DFS Similar
	\item FFT, NTT
	\item 多项式相关
\end{introduction}

\section{勾股数组}


- $m$能表示成$m=a^2+b^2$,且$gcd(a,b)=1$,当且仅当以下两个条件之一成立:

1. $m$是奇数,且$m$的每个素因子都模4余1

2. $m$是偶数,$m/2$是奇数且$m/2$的每个素因子都模4余1

\vbox{}

回顾一下,本原勾股数组

**定理2.1 (勾股数组定理).**  每个本原勾股数组(a,b,c)(a为奇数,b为偶数)都可从如下公式得出:
$$
a=st ,\quad  b=\frac{s^2-t^2}{2},\quad  c=\frac{s^2+t^2}{2}
$$
其中s>t>=1是任意没有公因数的奇数,即互质的奇数

\vbox{}

有以上两条定理可知,$c$是一个本原勾股数组的斜边当且仅当方程
$$
2c=s^2+t^2
$$
{\heiti 有互素的奇整数解$s,t$ }

且有如下命题

{\heiti 毕达哥拉斯斜边命题}  $c$是一个本原勾股数组斜边的充要条件是$c$ 是模4余1的素数的乘积。



\section{平方数之和、圆上整点}
这一节要解决的问题是给定一个数$x$,判断它能否分解为两个整数的平方之和,以及具体来说如何分解。从几何意义上考虑就是以原点为圆心,$\sqrt{x}$为半径的圆,是否经过坐标点(横纵坐标都是整数),
以及具体是哪些点。

先从$x$为素数考虑,这个时候规律相对简单。
\subsection{将素数分解为平方之和}

\begin{theorem}{定理}{primesquare}
	设$p$是素数,则$p$是两平方数之和的充要条件是$p\equiv 1 \ (mod \ 4)   $ 或$p=2$。
\end{theorem}

\begin{proof}
这里有两个断言:
\begin{itemize}
	\item $p$是两平方数之和;
	\item $p\equiv 1\ (mod\ 4)$ 或者 $p=2$。
\end{itemize}

要证明充要性有两个方面,即用一个断言作为条件去证明另外一个,下面分别证明。
\begin{enumerate}
	\item 若$p$是两平方数之和,则$-a^2\equiv b^2 (mod \ p)$ ,接着对两边取勒让德符号:(回顾第四章的内容)
	\begin{align*}
	\left(\frac{-1}{p}\right)\left(\frac{a}{p}\right)^2&=\left(\frac{b}{p}\right)^2 \\
	\left(\frac{-1}{p}\right)&=1
	\end{align*}
	于是由二次互反律知,$p$模4余1或者$p=2$。$\square$
	\item 当$p\equiv 1\ (mod\ 4)$或$p=2$时,要证明$p$一定可以表示成两平方数之和。也就是说只要我们能想到一种方法总能找到如何分解,就完成了证明。下面介绍一种方法,
	称之为{\heiti 费马降阶法(Fermat Descent Procedure)}。我们后面的代码就是基于此方法。
	
	我们从$A^2+B^2=Mp$开始,如果这里$M=1$,则证明完毕。所以我们考虑$M\ge 2$。
	费马的想法是,我们用现有的$A,B,M$,要是能构造出$a^2+b^2=mp \ \&\&\ m\le M-1$,
	则迭代下去,$m=1$时就完成了构造。

	在说具体的构造方法之前,先看一个恒等式,其正确性是显然的,在下面的构造过程中,这个恒等式起到关键作用。
	$$
	(u^2+v^2)(A^2+B^2)=(uA+vB)^2+(vA-uB)^2
	$$
	费马降阶法的过程如表\ref{tab:fermat-descent}所示:
	\begin{table}[!htbp]
		\centering
		\caption{费马降阶法 \label{tab:fermat-descent}}
		\begin{spacing}{1.5}
			\begin{tabular}{|c|c|}
				\toprule[1pt]
				举例(p=881) & 符号表示\quad p any prime $\equiv$ 1($mod\ 4$) \\
				\midrule[2pt]
				有$387^2+1^2=170*881$,$\ 170<881$ & 有$A^2+B^2=Mp$,$\ M<p$\\
				\midrule[1pt]
				\tabincell{c}{求得数$47,\ 1$使得\\$47\equiv 387\ (mod\ 170)$\\ $1\equiv 1\ (mod\ 170)$ \\ 其中$-\frac{170}{2}\le 47,1\le \frac{170}{2}$}  & 
				\tabincell{c}{求得数$u,\ v$使得\\$u\equiv A\ (mod\ M)$\\ $v\equiv B\ (mod\ M)$ \\ 其中$-\frac{M}{2}\le u,v\le \frac{M}{2}$}  \\
				\midrule[1pt]
				\tabincell{c}{于是有\\$47^2+1^2\equiv 387^2+1^2\equiv 0\ (mod\ 170)$}  & 
				\tabincell{c}{于是有\\$u^2+v^2\equiv A^2+B^2\equiv 0\ (mod\ M)$}  \\
				\midrule[1pt]
				\tabincell{c}{所以可写\\$47^2+1^2 = 170*13$ \\ $387^2+1^2 = 170*881$} & 
				\tabincell{c}{所以可写\\$u^2+v^2 = Mr(1\le r< M)$ \\ $A^2+B^2 = Mp$}  \\
				\midrule[1pt]
				\tabincell{c}{相乘可得\\$(47^2+1^2)(387^2+1^2) = 170^2*13*881$} & 
				\tabincell{c}{相乘可得\\$(u^2+v^2)(A^2+B^2) = M^2*r*p$}  \\
				\midrule[1pt]
				\multicolumn{2}{|c|}{利用恒等式$	(u^2+v^2)(A^2+B^2)=(uA+vB)^2+(vA-uB)^2 $} \\
				\midrule[1pt]
				\tabincell{c}{$(47*387+1*1)^2 + (1*387 - 47*1)^2$ \\ $=170^2*13*881$ \\ 所以$18190^2 + 340^2 = 170^2*13*881$} & 
				\tabincell{c}{$(uA+vB)^2+(vA-uB)^2 = M^2*r*p$}  \\
				\midrule[1pt]
				\tabincell{c}{两边同除以$170^2$得 \\ $(\frac{18190}{170})^2 + (\frac{340}{170})^2 = 13*881$ \\ $107^2 + 2^2 = 13*881$} & 
				\tabincell{c}{两边同除以$M^2$得 \\ $(\frac{uA+vB}{M})^2 + (\frac{vA-uB}{M})^2 = rp$ \\ (肯定可以整除) }  \\
				\midrule[1pt]
				\multicolumn{2}{|c|}{由此得到能表示成两平方数之和的$p$的更小倍数} \\
				\midrule[1pt]
				\multicolumn{2}{|c|}{重复上述过程,直到$p$本身能表成两平方数之和$(r=1)$} \\
				\bottomrule[1pt]
			\end{tabular}%
		\end{spacing}
	\end{table}%
	
	可以发现,每迭代一次,$p$的系数至少减半,即迭代次数为$O(logp)$次。为了说明上述过程的正确性,我们还要证明5个断言的正确性。
	\begin{enumerate}
		\item 一定可以找出数$A,B$,使得$A^2+B^2=Mp$,且$M<p$ 。
		\begin{proof}
			取同余式$x^2\equiv -1\ (mod \ p)$的一个解,由二次互反律知,当$p\%4=1$时,必定有解$x$。所以取
			$A=x,B=1$具有性质$p\ | \ A^2+B^2$,而且$M=\frac{A^2+B^2}{p}\le \frac{(p-1)^2+1^2}{p}<p$。
			$\square$ 
		\end{proof}
	\end{enumerate}


	在降阶程序的第二步,我们选取$u,v$使其满足
	$$
	u\equiv A (mod \ M),\quad v\equiv B(mod \ M), \quad -\frac{1}{2}M\le u,v\le \frac{1}{2}M
	$$
	于是有
	$$
	u^2+v^2\equiv A^2+B^2\equiv 0 \quad (mod \ M )
	$$
	设$u^2+v^2=rM$,其余四个断言如下:
	\begin{enumerate}[resume] % tell the enumerate to resume numbering
		\item $r\ge1$
		\item $r<M$ 
		\item $uA+vB$能被$M$整除
		\item $vA-uB$能被$M$整除
	\end{enumerate}
	这四个断言比较容易证明,这里不再给出。$\square$
\end{enumerate}
	至此,对定理\ref{thm:primesquare}的证明结束。$\square$
\end{proof}

\vbox{}

上面求解过程中关键的一步是求解$x^2\equiv -1 \ (mod \ p)$,可以直接使用二次剩余的模板,也可以使用随机算法。
即随机一个$a∈[1,p-1]$ ,求解$b\equiv a^{(p-1)/4}\ (mod\ p)$,由欧拉准则可知$b^2\equiv (\frac{a}{p}) \ (mod\ p)$,
即若选取的$a$不是$p$的二次剩余(有一半的概率),则求得的$b$即为解$x$。由二次剩余性质知有两个解$x_1,x_2$,且$x_1+x_2=p$。

代码如下:
\lstinputlisting[language=C++, style=codestyle2]{code07/random-algo-modsqr.cpp}

现在我们可以将素数分解为平方数之和了,下面看一下对于一般数该怎么做。

\subsection{将数分解为平方之和}
这一节的内容,不打算给出相关证明,而是给出一些结论和代码。因为我觉得下面这个视频已经讲的不错了,\href{https://www.bilibili.com/video/av12131743}{bilibili: 隐藏在素数规律中的$\pi$}。

\begin{theorem}{圆上整点数定理}{label}
	定义函数$\chi(n)$如下:
	\begin{align*}
		\chi(n) = \left\{\begin{matrix}
		1&  \quad  n\%4==1\\
		-1  & \quad  n\%4==3 \\
		0 &\quad   n\%4==0\ or\ 2
		\end{matrix}\right. \quad\quad n>0
	\end{align*}
	对于半径为$\sqrt{N}$的圆,圆上整点的数目(即将$N$分成两个平方数之和的方案数)可以这样计算:
	
	将$N$质因数分解:
	$$
	N = p_1^{k_1}*p_2^{k_2}*...*p_m^{k_m}
	$$
	则圆上整点数目 = $4*\Pi_{i=1}^{m}(\sum_{j=0}^{k_i}\chi(p_i^j))$。
	
	上面的这个式子只是为了统一,所以看起来规律不明显。实际上一句话:
	
	如果有模4余3的素数的指数为奇数答案为0,否则就是所有模4余1的素数的指数加1后乘起来最后再乘4。
\end{theorem}

\vbox{}

至于具体如何计算分解的方案,推荐大家看上面那个视频。大致思路就是对于可分解的素数利用费马降阶法进行分解,然后再在复数域中对不同质数
的结果组合计算一下。下面给出代码:

输入一个半径$r$,输出在圆上的所有整点。

{\heiti 时间复杂度:$O(A+B)$,其中$A$为质因子分解$r^2(thus\ r)$的时间,$B$为圆上整点数。

该代码在$r\le 10^9$时通过测试,更大时注意下会不会爆范围。}
\lstinputlisting[language=C++, style=codestyle2]{code07/Lattice-points.cpp}






\section{循环节问题}
\subsection{二阶常系数齐次线性递推循环节}
先将模数分解,在模素数幂意义下分别求循环节,最后取最小公倍数即可。而对于模素数幂有结论$G(p_i^{a_i}) = p_i^{a_1-1}G(p_i)$。

所以问题转为模素数如何处理。即给定$a,b,f(1),f(2)$,且满足$f(n)=a*f(n-1) + b*f(n-2)$。求$f(n)\ mod\ p$的循环节。

写成矩阵形式,如下:
$$
\left[\begin{array}{c}{f(n)} \\ {f(n-1)}\end{array}\right]=\left[\begin{array}{cc}{a} & {b} \\ {1} & {0}\end{array}\right]\left[\begin{array}{c}{f(n-1)} \\ {f(n-2)}\end{array}\right]
$$

变形一下:
$$
\left[\begin{array}{c}{f(n+2)} \\ {f(n+1)}\end{array}\right]=\left[\begin{array}{cc}{a} & {b} \\ {1} & {0}\end{array}\right]^{n}\left[\begin{array}{c}{f(2)} \\ {f(1)}\end{array}\right]
$$
那么现在的问题就转化为求最小的$n$,使得
$$
\left[\begin{array}{ll}{a} & {b} \\ {1} & {0}\end{array}\right]^{n}(\bmod p)=\left[\begin{array}{ll}{1} & {0} \\ {0} & {1}\end{array}\right]
$$
结论:设$c=a^2+4b$, 有两种情况:
\begin{enumerate}
	\item 若$c$是模$p$的二次剩余,则$n$是$p-1$的因子;
	\item 若$c$不是模$p$的二次剩余,则$n$是$(p+1)(p-1)$的因子
\end{enumerate}
所以只要枚举因子并判断即可。
时间复杂度$O(T*2^3*log(p))$,其中$T$为$p-1$或$(p+1)(p-1)$的因子数目。

\subsection{分数小数的循环节}
\begin{custom}{问题}
给定一个分数$\frac{p}{q}$,将其转化为小数。
\end{custom}

\begin{solution}
考虑一个分数,有以下三种情况:
\begin{itemize}
	\item 整数
	\item 有限小数
	\item 无限循环小数
\end{itemize}

先来考虑下无限循环小数,我们先将$p$和$q$公共的因子除去,然后利用循环这个性质,可知存在$i,j$满足,$p*10^i \equiv p*10^j \ (mod \ q), \ where\ j>i\ge0$。
{\heiti 这里$i$代表循环出现前有多少位数,$j-i$表示循环节长度。}
化简后得$p*10^i*(10^{j-i}-1) \equiv 0\   (mod \ q)$。也就是说,$q\ |\ p*10^i*(10^{j-i}-1)$。由于$10^{j-i}-1$中一定不存在因子$2$和$5$,所以要想是$q$的倍数,
$2$和$5$的贡献均来自$10^i$。于是我们记录$q$中$2$的次幂数为$num2$,$5$的次幂数为$num5$,则$i = max(num2,num5)$。记$q$除去所有$2$和$5$因子后,为$m$,则$10^{j-i}\equiv 1 \ (mod\ m)$。
然后我们解同余方程$10^x\equiv 1\ (mod\ m),\ where\ x>0$,$j$就等于$x+i$。至于这个方程,$x$一定是$\varphi(m)$的因子,枚举因子判定即可。

如果$j$无解,表示该小数为有限小数,$i$也就表示了小数点后共有几位;如果$j$有解,含义如上。
\end{solution}

\href{https://www.luogu.org/problem/P1530}{luogu P1530 分数化小数}

时间复杂度:$\sqrt{q}*log(q)$ \quad (确定i, j)

输出格式: 按照下面规则,如果结果长度大于76,每行输出76个字符。
\begin{itemize}
	\item 2/2 = 1.0
	\item 3/8 = 0.375
	\item 45/56 = 0.803(571428)
\end{itemize}

\lstinputlisting[language=C++, style=codestyle2]{code07/fraction2decimal.cpp}

\begin{note}
	由于上面这个题目要求输出时每行只要$76$个字符,我在写代码时一开始没有注意,因此直接使用的cout。所以代码中我使用了下面的方法:
	\lstinputlisting[language=C++, style=codestyle2]{code07/redirect-cout.cpp}
	
	先将cout重定向到stringstream,然后从其中取出string,最后将cout还原到stdout。
\end{note}

\section{DFS Similar}
dfs similar是指一类可以“暴力”搜索的问题,这类问题往往有很好的性质使得暴力的时间不会太长。

\subsection{Counting Sequences I(18上海网络赛D)}
\begin{custom}{问题}
我们定义一个由正整数组成的序列$a_1,a_2,...,a_n$是好的当:
\begin{itemize}
\item $n\ge2$
\item $a_1+...+a_n = a_1*...*a_n$
\end{itemize}
请输出有多少种这样的序列,$2\le n \le 3000$。
\end{custom}




\subsection{反欧拉函数(洛谷P4780)}
\begin{custom}{问题}
求最小的正整数$x$,使得$\phi(x)=n$。输出$x$,如果$x>2^{31}$或者不存在,则输出$-1$。
\end{custom}




\subsection{因子个数最多的数(51nod1060)}
\begin{custom}{问题}
把一个数的约数个数定义为该数的复杂程度,给出一个n,求$1\sim n$中复杂程度最高的那个数。如果有多个数复杂度相等,输出最小的。
$1\le n\le 10^{18}$。
\end{custom}

即给定$N$,求$[1,N]$之间最大的反素数**(即拥有因子数目最多的数)**

性质:
\begin{itemize}
	\item 一个反素数的质因子必然是从2开始连续的质数。 
	\item $p=2^{t_1} * 3^{t_2} * 5^{t_3} * 7^{t_4}$...必然t1>=t2>=t3>=...
\end{itemize}

暴力dfs    时间复杂度大概几个log

\lstinputlisting[language=C++, style=codestyle2]{code07/mostfactors.cpp}





\section{FFT 与 NTT}

\subsection{FFT}

\subsection{NTT}

\section{多项式}

\subsection{拉格朗日插值法}

\subsection{多项式求逆}

\subsection{多项式取模}

\subsection{多项式开方}

\subsection{多项式多点求值}

\subsection{多项式多点插值}



\begin{problemset}
	\item xx
\end{problemset}


\nocite{*} 

\bibliography{reference}